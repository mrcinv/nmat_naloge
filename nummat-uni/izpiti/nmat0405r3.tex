

\begin{center}
  {\large Re"sitve drugega kolokvija iz UVNM\\
    31.1.2005\\
    }
\end{center}

\begin{enumerate}

  \item Naj bo $A\in\RR^{n\times n}$ tridiagonalna matrika. 
    Napi"site algoritem, ki v "casu ${\cal O}(n)$ izra"cuna
    $LU$ razcep matrike $A$. Utemeljite, zakaj je va"s
    algoritem res ustrezne "casovne zahtevnosti.\\
    Predlagani algoritem uporabite za izra"cun $LU$ razcepa matrike
    $$A=\left(
      \begin{array}{ccc}
        1 & 2 & 0\\
        1 & 3 & 2\\
        0 & 1 & 3
      \end{array}
      \right).$$
   {\bf Re"sitev}: Ker je matrika tridiagonalna, lahko pri $LU$
   razcepu upo\-"ste\-va\-mo njeno strukturo in ne izvajamo odve"cnih operacij
   (mno"zenj z ni"clo). Algoritem zapi"semo v Matlabovi obliki takole:
   \begin{verbatim}
     function [L,U]=lu3diag(A);
       n=size(A,1); %stevilo vrstic matrike A
       L=eye(n); %identiteta dim. nxn
       U=triu(A);%U=A na zacetku!
       for i=2:n
           L(i,i-1)=A(i,i-1)/U(i-1,i-1);%kvocient
           U(i,i)=U(i,i)-L(i,i-1)*U(i-1,i);%eliminacija
       end
   \end{verbatim}
   Ker je v algoritmu ena sama zanka dol"zine $n-1$, znotraj nje
   pa na vsakem koraku konstantno mnogo operacij, je "casovna
   zahtevnost ${\cal O}(n)$.\\
   Za matriko $A$ dobimo
   $$L=\left(
         \begin{array}{ccc}
           1 & 0 & 0\\
           1 & 1 & 0\\
           0 & 1 & 0
         \end{array}
      \right) \quad {\rm in} \quad 
      U=\left(
         \begin{array}{ccc}
           1 & 2 & 0\\
           0 & 1 & 2\\
           0 & 0 & 1
         \end{array}
      \right)$$
  
  \item Tabelo podatkov
   $$\begin{array}{c|rrrrr}
     x_i & -2 & -1 & 0 & 1 & 2\\ \hline
     y_i &  2 &  1 & 0 & 1 & 2
     \end{array}
   $$
   zaradi o"citne simetrije aproksimiramo s parabolo oblike
    $p(x)=\alpha\,x^2+\beta$. 
    Izra"cunajte tisti par parametrov $\alpha$ in
    $\beta$, pri katerem je dose"zena vrednost
    $$\min_{\alpha,\beta\in\RR}\sum_{i=1}^5(p(x_i)-y_i)^2.$$
    {\bf Re"sitev}: Opazimo, da gre za klasi"cno metodo najmanj"sih 
    kvadratov, z aproksimacijsko funkcijo $p(x)=\alpha\,x^2+\beta$.
    Lahko re"sujemo preko normalnega sistema ena"cb (vemo, v praksi bi
    uporabili $QR$ razcep). Matrika normalnega sistema je torej
    $M=A^TA$, kjer je 
    $$A=\left(
             \begin{array}{cc}
               x_1^2 & 1\\
               x_2^2 & 1 \\
               \vdots & \vdots \\
               x_5^2 & 1
             \end{array}
      \right).
      $$
      Torej je 
      $$M=\left(
                   \begin{array}{cc}
                     34 & 10\\
                     10 & 5
                   \end{array}
      \right),
      $$
      koeficienta $\alpha$ in $\beta$ pa re"sitev sistema
      $M[\alpha,\beta]^T=A^T[y_1,\dots,y_5]^T=[18,6]^T$.
      Dobimo $\alpha=0.4286$ in $\beta=0.3429$.
      
      
  \item Naj bo 
    $$ 
    A=\begin{pmatrix}
      1   & 0.1 & 0.2\cr
      0.1 &  2  & -0.1\cr
      x   &  0.2& 3
      \end{pmatrix}.
    $$
    Naslednje probleme re"site {\bf brez ra"cunanja} lastnih vrednosti:
    \begin{itemize}
      \item[a)] Preverite, da matrika $A$ za $x\in(-2.8,2.8)$ ne more imeti 
        negativnih lastnih vrednosti. 
      \item[b)] Za kak"sne $x\in\RR$ gotovo nima nobene dvojne
        lastne vrednosti? 
      \item[c)] "Cim bolje ocenite najmanj"so lastno vrednost v primeru, 
        ko je $x=0$.
    \end{itemize}
    {\sl Namig: Uporabite Gershgorinov izrek.}\\
    {\bf Re"sitev}: Po Gersgorinovem izreku je vsaka lastna vrednost
    	matrike $A\in \RR^{n\times n}$ vsebovana v uniji krogov
    	$K_i$, kjer je 
    	$$K_i=\{z\in\CC;\ |z-a_{ii}|\leq \sum_{j=1,j\neq i}^n |a_{ij}|.$$
    	Velja "se ve"c, "ce je kak"sen krog izoliran od ostalih, je v njem
    	natanko ena lastna vrednost.\\
    	V na"sem primeru so radiji krogov $R_1=|0.1|+|0.2|=0.3$,
    	$R_2=|0.1|+|-0.1|=0.2$ in $R_3=|x|+|0.2|$.
    	"Ce ozna"cimo s $K(s,r)$ krog s sredi"s"cem v $(s,0)$ in radijem $r$,
    	potem so vse lastne vrednosti v uniji krogov
    	$K(1,0.3)$, $K(2,0.2)$ in $K(3,|x|+0.2)$. "Ce je torej $|x|\leq 2.8$,
    	v uniji ne more biti "stevil z negativno realno komponento.\\
    	Matrika zagotovo nima dvojne lastne vrednosti, "ce so vse tri lastne
    	vrednosti med seboj razli"cne. To pa je gotovo res tedaj, ko se
    	Gersgorinovi krogi ne sekajo, torej, "ce je $|x|<0.6$.\\
    	Pri izbolj"sanju ocene za lastne vrednosti se lahko opremo
    	na dejstvo, da imata matriki $A$ in $X^{-1}AX$ iste lastne vrednosti.
    	"Se posebej je to dejstvo zanimivo tedaj, ko za $X$ izberemo 
    	kak"sno preprosto matriko (na primer diagonalno).\\
    	Vzemimo torej $X={\rm diag}(a,1,1)$, kjer je $a>0$.
    	V matriki $X^{-1}AX$ se prva vrstica mno"zi z $1/a$, prvi
    	stolpec pa z $a$. Torej moramo paziti, da se prvi in drugi
    	krog ne zdru"zita (tretjemu se radij ne spremeni, saj je
    	$a_{31}=x=0$). Z malo razmisleka opazimo, da se
    	prvi in drugi krog ne bosta zdru"zila, "ce bo
    	$$ \frac{0.1}{a}+\frac{0.2}{a}+0.1a+0.1<1.$$
    	To prevedemo nakvadratno neena"cbo $a^2-9a+3<0$.
    	Zato mora biti $a$ manj"si od ve"cjega korena kvadratne ena"cbe
    	$a^2-9a+3=0$, torej
    	$$a<\frac{9+\sqrt{69}}{2}=8.6533$$
    	Pri temu izbranemu $a$ je prvi radij manj"si od 
    	$0.3/a=0.0347$, torej je najmanj"sa lastna vrednost
    	na intervalu $(0.9653,1.0347)$.
    	
  
  \item Re"sujete {\sl Besselovo} diferencialno ena"cbo
    $$x^2\,y''(x)+x\,y'(x)+x^2\,y(x)=0, \quad y(0)=1,\ y'(0)=0.$$
    Z Eulerjevo metodo in korakom $h=0.1$ ocenite vrednost
    $y'(0.2)$.\\
    {\sl Namig: $y''(0)=-\frac{1}{2}$}.\\
    {\bf Re"sitev}: Ena"cbo drugega reda prevedemo na sistem prvega
    $$\bfm{Y}'=[y',-\frac{1}{x}y'-y]^T=\bfm{F}(x,\bfm{Y}).$$
    Z Eulerjevo metodo tako dobimo
    $$\bfm{Y}_1=\bfm{Y}_0+h\bfm{F}(0,\bfm{Y}_0)=[1,0]^T+0.1[0,-1/2]^T=[1,-0.05]^T$$
    (tu smo upo"stevali namig, da je $y''(0)=-1/2$, saj zaradi deljenja z $0$
    ne moremo izra"cunati $y''(0)$).
    Na drugem koraku dobimo
    \begin{eqnarray*}
    \bfm{Y}_2&=&\bfm{Y}_1+h\bfm{F}(0.1,\bfm{Y}_1)\\
     &=&[1,-0.05]^T
    +0.1[-0.05,-(1/0.1)(-0.05)-1]^T
    = [0.995,-0.1]^T.
    \end{eqnarray*}
    Pribli"zek za $y'(0.2)$ je torej $-0.1$.

\end{enumerate}
