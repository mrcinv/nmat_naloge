

\begin{center}
  {\large IZPIT IZ NUMERI"CNE MATEMATIKE\\
    25.8.2006}
\end{center}
\vspace{1.5cm}
\begin{enumerate}
\item Tabelo podatkov
  $$
    \begin{array}{r|llll}
    	x & -3 & -2 & -1 & 0 \\ \hline
    	f(x)& -23 & -5 &  1 &    1
    \end{array}
  $$
  interpolirate s kubi"cnim polinomom.
      \begin{enumerate}
        \item Zapi"site interpolacijski polinom v Newtonovi obliki. 
        \item Newtonovo obliko iz to"cke a) prepi"site
          v standardno (torej v obliko $p_3(x)=a_3\,x^3+
          a_2\,x^2+a_1\,x+a_0$).
        \item Predpostavite, da tabela predstavlja vrednosti
          neznane zvezne funkcije $f$ na $[-3,0]$.
          Utemeljite, da ima $f$ ni"clo na $[-2,-1]$. 
          Ni"clo ocenite tako, da namesto ni"cle funkcije $f$ (ki je
          seveda ne poznate) izra"cunate ni"clo interpolacijskega
          polinoma, ki ste ga zapisali v to"cki b). 
          Ni"clo dolo"cite na dve decimalni mesti s katerokoli
          numeri"cno metodo.
      \end{enumerate}
\item Re"sujete za"cetni problem drugega reda
  $$
    y''+2\,x\,y'+2\,x\,y=(3\,x-2)e^{-x},\quad y(0)=0,\quad y'(0)=1.
  $$
  \begin{enumerate}
    \item Prevedite zgornji problem na re"sevanje
      sistema dveh diferencialnih ena"cb prvega reda.
    \item Z Eulerjevo metodo in korakom $h=0.1$ 
      izra"cunajte numeri"cni pribli"zek za re"sitev
      sistema v to"cki $x=0.2$. 
    \item Preverite, da je
      $y(x)=x\,e^{-x}$ re"sitev za"cetnega problema 
      (iz\-ra\-"cu\-naj\-te
      odvoda, vstavite v diferencilano ena"cbo in primerjajte
      levo in desno stran).
      Kolik"sna je torej absolutna napaka pribli"zka za $y'(0.2)$
      iz to"cke b)?
  \end{enumerate}

\end{enumerate}


