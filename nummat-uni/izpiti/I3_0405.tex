\documentclass[11pt,a4paper]{article}

\usepackage[slovene]{babel}
\usepackage{amsfonts,amsmath}
%Real and complex numbers
\def\RR{\mathbb{R}}
\def\CC{\mathbb{C}}
\def\NN{\mathbb{N}}
% Boldface math
\def\bfm#1{{\dimen0=.01em\dimen1=.009em\makebold{$#1$}}}

\def\makebold#1{\mathord{\setbox0=\hbox{#1}
       \copy0\kern-\wd0
       \raise\dimen1\copy0\kern-\wd0
       {\advance\dimen1 by \dimen1\raise\dimen1\copy0}\kern-\wd0
       \kern\dimen0\raise\dimen1\copy0\kern-\wd0
       {\advance\dimen1 by \dimen1\raise\dimen1\copy0}\kern-\wd0
       \kern\dimen0\raise\dimen1\copy0\kern-\wd0
       {\advance\dimen1 by \dimen1\raise\dimen1\copy0}\kern-\wd0
       \kern\dimen0\raise\dimen1\copy0\kern-\wd0
       \kern\dimen0\box0}}


\pagestyle{empty}

\begin{document}

\begin{center}
  {\large UVOD V NUMERI"CNE METODE\\
    3. izpit, 13.6.2005\\
    }
\end{center}
\vspace{1cm}

\begin{enumerate}
  
  \item Integral $I=\int_{a}^{a+h} f(x)\,dx$ ra"cunate z enostavnim
      trapeznim pravilom s korakom $h$, torej
      $$ I\approx T(h):=\frac{h}{2}\left(f(a)+f(a+h)\right).$$
      Kako bi izra"cunali $h$, "ce poznate $T(h)$?
      Naj bo $a=0$, $f(x)=\exp(-x)$ in $T(h)=0.4016$. Koliko je v tem
      primeru $h$?

  \item Temperatura v razgretem avtomobilu se spreminja s 
    "casom pribli"zno takole
    $$T(t)=T_0+\Delta T_0\,\exp(k\,t),$$
    kjer je $T(t)$ temperatura ob
    "casu $t$ v notranjosti avtomobila, $T_0$ znana konstantna zunanja
    temperatura, $\Delta T_0$ razlika med notranjo in zunanjo
    temperaturo ob za"cetku hlajenja (torej ob "casu $t=0$) in $k$
    parameter, ki opisuje hitrost hlajenja.\\
    Zunanja temperatura je $T_0=30C$. Temperaturo v 
    avtomobilu ste izmerili v minutnih intervalih in dobili naslednjo
    tabelo:%T0=30;k=-0.15/min;dT_0=15;
    $$
      \begin{array}{c|ccccc}
        t & 1 & 2 & 3 & 4 & 5\\ \hline
        T & 42.9106 & 41.1123 & 39.5644 & 38.2322 & 37.0855
      \end{array}.
    $$
    Z metodo najmanj"sih kvadratov dolo"cite $\Delta T_0$ in $k$ za ta primer.\\
    {\sl Namig: Logaritmirajte koli"cino $T(t)-T_0$ in uporabite klasi"cno
    metodo najmanj"sih kvadratov na logaritmiranih podatkih.}
        
  
  \item Za matriko $A\in\RR^{n\times n}$ poznamo njen $LU$ razcep,
        torej $A=L\,U$. Zapi"site {\bf u"cinkovit} algoritem za
            re"sevanje sistema $A^T\,\bfm{x}=\bfm{b}$, kjer
            je $\bfm{b}\in\RR^n$ znani vektor desnih strani sistema.
            Kolik"sna je "casovna zahtevnost algoritma?
            Izra"cunajte $\bfm{x}$, "ce je
            $$
              A=L\,U=
              \left[
                \begin{array}{ccc}
                  1 & 0 & 0\\
                  1 & 1 & 0\\
                  1 & 1 & 1
                \end{array}
               \right]
               \left[
                \begin{array}{ccc}
                  1 & 2 & 3\\
                  0 & -1 & 2\\
                  0 & 0 & -1
                \end{array}
               \right]
            $$
            in $\bfm{b}=[6,7,25]^T$.
   \item Za"cetni problem 
     $$y^{(2005)}(x)-2\,x\,y=0,\quad y(1)=1,\,y'(1)=1,\,\dots, y^{(2004)}(1)=1,
     $$
     re"sujete z Eulerjevo metodo s korakom $h=0.1$. Kak"sen je pribli"zek
     za $y^{(2004)}(0.2)$.

\end{enumerate}
\end{document} 
