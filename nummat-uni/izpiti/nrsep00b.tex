\documentclass[12pt,a4paper]{article}

\usepackage[slovene]{babel}

\pagestyle{empty}

\begin{document}
\begin{center}
  IZPIT IZ NUMERI"CNE MATEMATIKE\\
  21. september 2000
\end{center}

\begin{enumerate}

  \item Re"sujemo ena"cbo
    $$f(x)=10\,\sin{x}-x\,\int_{2}^{3}\frac{t}{\log{t}}\,dt=0.$$
  \begin{enumerate}
   
     \item Integral v predpisu funkcije $f$ izra"cunaj s trapeznim
       pravilom s korakom $h=0.25$.

     \item Izra"cunaj $f(1)$ in $f(3)$ ter se prepri"caj, da 
       ima zgornja ena"cba vsaj eno re"sitev na intervalu
       $[1,3]$.

     \item Zapi"si ekonomi"cen algoritem, ki ra"cuna re"sitev
       zgornje ena"cbe na intervalu $[1,3]$ po metodi regula 
       falsi.

     \item Dolo"ci re"sitev na omenjenem intervalu z metodo 
       regula falsi na dve decimalni mesti natan"cno.
  \end{enumerate}

  \item Re"sujemo za"cetni problem
    $$y'=3\,\frac{y}{x}-x,\quad y(1)=1.$$

  \begin{enumerate}
    
    \item Poi"s"ci to"cno re"sitev za"cetnega problema.

    \item Z Eulerjevo metodo izra"cunaj pribli"zno re"sitev
      na intervalu $[1,1.5]$ z dol"zino koraka $h=0.25$. Kolik"sna
      je globalna napaka pri $x=1.5$?

    \item Izra"cunaj pribli"zno re"sitev "se z eno od metod 
      Runge-Kutta drugega reda (glej Osnove numeri"cne matematike, str. 146)
      in enakim korakom kot v to"cki (b). Kolik"sna je sedaj globalna 
      napaka pri $x=1.5$?

    \item {\it Dodatek}: Kolik"sna je v obeh primerih lokalna 
      napaka pri $x=1.5$?

  \end{enumerate} 
\end{enumerate}


\end{document}
         