\documentclass[12pt,a4paper]{article}

\usepackage[slovene]{babel}
\usepackage{amsfonts}
\pagestyle{empty}

\begin{document}

  \begin{center}
    {\Large IZPIT IZ NUMERI"CNE MATEMATIKE}\\
    6.9.2000
  \end{center}
\begin{enumerate}

  \item Naj bo $A$ tridiagonalna kvadratna simetri"cna pozitivno definitna 
  matrika.   

    \begin{enumerate}
       
    \item Zapi"si ekonomi"cen algoritem za $LU$ razcep take matrike
      in pre"stej "stevilo potrebnih mno"zenj in deljenj (pivotiranje ni 
      potrebno).
    \item Zapi"si ekonomi"cen algoritem za razcep Choleskega take matrike
      in pre"stej "stevilo potrebnih mno"zenj in deljenj. Korenjenje "stej
      za "stiri mno"zenja.
    \item Re"si sistem linearnih ena"cb 
      \begin{displaymath}
        \left[
        \begin{array}{rrr}
          4&-1&0\\
          -1&4&-1\\
          0&-1&4
        \end{array}
        \right]\,\mbox{\boldmath{$x$}}=
        \left[
        \begin{array}{c}
          2\\ 4\\ 10
        \end{array}
        \right]
      \end{displaymath}
    z razcepom Choleskega. 
    
    \end{enumerate}

  \item \textbf{Kvadratna} funkcija $f$ je podana s tabelo vrednosti
    \begin{displaymath}
      \begin{array}{c|ccc}
        x&1&2&3\\
        \hline
        f(x)&1.86&5.22&11.34
      \end{array}
    \end{displaymath}
        
    \begin{enumerate}
    
    \item Sestavi tabelo deljenih diferenc.

    \item Izra"cunaj $f[1,2,3,4]$.     
           
    \item Izra"cunaj $f(1.41)$.
          

    \end{enumerate}

\end{enumerate}
\end{document}
         