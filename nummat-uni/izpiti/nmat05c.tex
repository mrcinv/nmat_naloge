

\begin{center}
  {\large IZPIT IZ NUMERI"CNE MATEMATIKE\\
    29.8.2005}
\end{center}
\vspace{1.5cm}
\begin{enumerate}
  \item Robot se giblje po krivulji, ki je podana parametri"cno
        s predpisom
        $$\mathbf{r}(t)=
        \left(
          \begin{array}{c}
            t\\
            e^{-t}
          \end{array}
        \right),\quad t\geq 0.
        $$
        V to"cki $\mathbf{T}(2,2)$ se nahaja kontrolni senzor. 
        Izra"cunajte tisto to"cko na krivulji gibanja 
        robota, v kateri je robot najbli"zje senzorju.. 
  \begin{enumerate}
    \item Razdaljo robota do senzorja izrazite kot funkcijo
      $f(t)$.
    \item I"s"cete torej minimum funkcije $f$. Vsi kandidati za 
      minimum so med re"sitvami nelinearne 
      ena"cbe $f'(t)=0$. Zapi"site jo!
    \item Ena"cba $f'(t)=0$ ima eno samo re"sitev. Poi"s"cite jo na
      tri decimalna mesta natan"cno z Newtonowo metodo. Nato
      dolo"cite iskano to"cko na krivulji.
  \end{enumerate}
  \item Integral funkcije $f$ na intervalu $[0,h]$ "zelite pribli"zno
    ra"cunati s formulo
    $$\int_0^h f(x)\,dx\approx \frac{h}{2}\,
    \left(f(x_1\,h)+f(x_2\,h)\right).$$
  \begin{enumerate}
    \item Dolo"cite "stevili $x_1$ in $x_2$ tako, da bo formula
      "cim vi"sjega reda.
    \item Z dobljeno formulo pribli"zno izra"cunajte integral
      $$\int_0^{0.5} \sin{x}\,dx.$$
    \item Nato integral iz to"cke (b) izra"cunajte "se s trapezno 
      formulo in ugotovite, katera formula je v tem primeru
      bolj natan"cna.
  \end{enumerate}
\end{enumerate}


