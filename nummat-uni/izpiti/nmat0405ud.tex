\documentclass[12pt,a4paper]{article}

\usepackage[slovene]{babel}
\usepackage{amsfonts,amsmath}
%Real and complex numbers
\def\RR{\mathbb{R}}
\def\CC{\mathbb{C}}
\def\NN{\mathbb{N}}
% Boldface math
\def\bfm#1{{\dimen0=.01em\dimen1=.009em\makebold{$#1$}}}

\def\makebold#1{\mathord{\setbox0=\hbox{#1}
       \copy0\kern-\wd0
       \raise\dimen1\copy0\kern-\wd0
       {\advance\dimen1 by \dimen1\raise\dimen1\copy0}\kern-\wd0
       \kern\dimen0\raise\dimen1\copy0\kern-\wd0
       {\advance\dimen1 by \dimen1\raise\dimen1\copy0}\kern-\wd0
       \kern\dimen0\raise\dimen1\copy0\kern-\wd0
       {\advance\dimen1 by \dimen1\raise\dimen1\copy0}\kern-\wd0
       \kern\dimen0\raise\dimen1\copy0\kern-\wd0
       \kern\dimen0\box0}}


\pagestyle{empty}

\begin{document}

\begin{center}
  {\large UVOD V NUMERI"CNE METODE\\
    4. izpit, 12.9.2005\\
    }
\end{center}
\vspace{1cm}

\begin{enumerate}
  
  \item Delec se giblje po krivulji
    \begin{equation*}
    	\bfm{r}(t)=
    	\begin{pmatrix}
    	  e^{-t}\\
    	  t
    	\end{pmatrix}, \quad t\geq 0.
    \end{equation*}
    Numeri"cno izra"cunajte tisto to"cko na krivulji, v
    kateri je delec najbli"zje to"cki $\bfm{T}(0,1)$.
    Rezultat naj bo to"cen na dve decimalni mesti.
    
  \item Integral funkcije $f$ na intervalu $[0,h]$
    ra"cunate po pribli"zni formuli
    \begin{equation*}
      \int_0^h f(x)\,dx\approx \frac{h}{2}
      \left(f(x_1)+f(x_2)\right).
    \end{equation*}
    Dolo"cite "stevili $x_1$ in $x_2$ tako, da bo
    formula "cim vi"sjega reda. Ali ju lahko dolo"cite
    tako, da bo formula natan"cna za polinome stopnje
    $\leq 3$? Z izpeljano formulo numeri"cno
    izra"cunajte 
    \begin{equation*}
      \int_0^{0.3}\sin{\sqrt{x}}\,dx.
    \end{equation*}
  
  \item Naj bo 
    \begin{equation*}
      A=
      \begin{pmatrix}
        x & 1\\
        1 & x
      \end{pmatrix},\quad x>1.
    \end{equation*}
    Dolo"cite ${\rm cond}_\infty(A)$ in
    ${\rm cond}_1(A)$.
    
  \item Re"sujete za"cetni problem 
    \begin{equation*}
      y^{'}=x\,y,\quad y(0)=1.
    \end{equation*}
    Kolik"sen je lahko najve"c korak $h>0$, da bo 
    absolutna napaka pribli"zka po enem koraku Eulerjeve metode
    manj"sa kot $0.1$.\\
    {\sl Namig: Najprej izra"cunajte to"cno re"sitev!}
\end{enumerate}
\end{document} 
