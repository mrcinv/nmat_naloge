

\begin{center}
  {\large IZREDNI IZPIT IZ NUMERI"CNE MATEMATIKE\\
    28.11.2005}
\end{center}
\vspace{1.5cm}
\begin{enumerate}
  %nelinearne
  \item Naj bo $a>0$. Z navadno iteracijo oblike
    $$x_{n+1}=\alpha\,x_n+\beta\,\frac{a}{x_n},\quad n=0,1,\dots$$
    "zelite izra"cunati pribli"zek $\sqrt{a}$. 

  \begin{enumerate}
    \item Dolo"cite parametra 
    $\alpha$ in $\beta$ tako, da bo iteracija "cim hitreje
    konvergirala za dovolj dober za"cetni pribli"zek.

    \item Prepri"cajte se, da je iteracija, ki ste jo dobili v a),
      ravno Newtonova metoda za re"sevanje ena"cbe $x^2-a=0$.

    \item Z zgoraj izra"cunano iteracijo izra"cunajte
      $\sqrt{10}$ na "stiri decimalna mesta natan"cno. Za $x_0$ izberite
      $3$.

  \end{enumerate}
  {\sl Nasvet: Iteracijo zapi"site v obliki $x_{n+1}=g(x_n)$ in
    zahtevajte, da je $g(\sqrt{a})=\sqrt{a}$ ter $g'(\sqrt{a})=0$.}
  %interpolacija
  \item Za neznano funkcijo $f$ poznate naslednje podatke
    $$
    \begin{array}{c|rrr}
      x& -1 & 0 & 1 \\ \hline
      f(x) & 1 & 1 & -1
    \end{array}.
    $$
      

  \begin{enumerate}
    \item Izra"cunajte Newtonov intepolacijski polinom 
      za zgornjo tabelo.

    \item S pomo"cjo polinoma iz to"cke a) ocenite, kje ima $f$ ni"clo. 

    \item Ocenite "se, kje ima $f$ lokalni maksimum (spet
      s pomo"cjo polinoma).

  \end{enumerate}
  {\sl Namig: Ni"cla in ekstrem $f$ sta pribli"zno tam, kjer sta 
    ni"cla in ekstrem polinoma.}
\end{enumerate}

