\documentclass[12pt,a4paper]{article}

\usepackage[slovene]{babel}

\def\bfm#1{{\dimen0=.01em\dimen1=.009em\makebold{$#1$}}}
 
\def\makebold#1{\mathord{\setbox0=\hbox{#1}%
       \copy0\kern-\wd0%
       \raise\dimen1\copy0\kern-\wd0%
       {\advance\dimen1 by \dimen1\raise\dimen1\copy0}\kern-\wd0%
       \kern\dimen0\raise\dimen1\copy0\kern-\wd0%
       {\advance\dimen1 by \dimen1\raise\dimen1\copy0}\kern-\wd0%
       \kern\dimen0\raise\dimen1\copy0\kern-\wd0%
       {\advance\dimen1 by \dimen1\raise\dimen1\copy0}\kern-\wd0%
       \kern\dimen0\raise\dimen1\copy0\kern-\wd0%
       \kern\dimen0\box0}}   


\pagestyle{empty}

\begin{document}
\begin{center}
  IZPIT IZ NUMERI"CNE MATEMATIKE\\
  6. september 2001
\end{center}

\vspace{1.5cm}
\begin{enumerate}
 \item Re"sujemo sistem linearnih ena"cb
    \begin{eqnarray*}
      x_1+x_2 &=& 1\\
      x_2+x_3 &=& 2\\
              &\vdots&\\
      x_n+x_1 &=& n,
    \end{eqnarray*}
    kjer je $n$ liho "stevilo.
 \begin{itemize}
    \item[a)] Zapi"site sistem v matri"cni obliki in doka"zite,
      da je vedno re"sljiv.
    \item[b)] Predlagajte algoritem za re"sevanje tega sistema in
      pre"stejte "stevilo potrebnih operacij (mno"zenj in deljenj).
      Algoritem naj bo kar se da u"cinkovit.
    \item[c)] Re"site sistem v primeru $n=5$.
 \end{itemize}

 \item Radi bi izra"cunali nekaj decimalk "stevila 
        $\sqrt{2}$. V ta namen re"sujemo ena"cbo $x^2-2=0$
        z navadno iteracijo takole
        $$x_{n+1}=x_n+c\,(x_n^2-2).$$
 \begin{itemize}
      \item[a)] Izra"cunajte prve "stiri pribli"zke, "ce je
        $c=-0.3$ in $x_0=1$.
      \item[b)] Za katere vrednosti parametra $c$ metoda 
        konvergira, "ce vzamemo $x_0$ dovolj blizu $\sqrt{2}$?
      \item[c)] Dolo"cite parameter $c$, pri katerem metoda 
        najhitreje konvergira.
 \end{itemize}
\end{enumerate}
\end{document}
         
