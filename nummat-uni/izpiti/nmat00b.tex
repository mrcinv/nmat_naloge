

\begin{center}
  {\Large {IZPIT IZ NUMERI"CNE MATEMATIKE}}\\
   28. junij 2000
\end{center}

\vspace{1cm}

\begin{enumerate}

  \item Pribli"zke za re"sitev nelinearne ena"cbe $f(x)=0$ 
  ra"cunamo  s {\sl Halleyejo metodo}:
  $$x_{n+1}=x_n-\frac{f(x_n)}{f'(x_n)-\frac{1}{2}f''(x_n)
  \frac{f(x_n)}{f'(x_n)}}.$$

  \begin{enumerate}
   
   \item Napi"si "cim bolj ekonomi"cen algoritem ("cim manj izra"cunov 
   vrednosti funkcije in odvodov) za ra"cunanje
   pribli"zkov po Halleyevi metodi.

   \item Izra"cunaj prva dva pribli"zka v primeru, ko je
   $x_0=2$ in $f(x)=x^3-2$. Kam konvergira zaporedje pribli"zkov,
   "ce je konvergentno? Koliko se $x_2$ razlikuje od re"sitve 
    ena"cbe?
    

   \item Izra"cunaj prva dva pribli"zka pri enakih pogojih kot v (b)
   "se z Newtonovo metodo in primerjaj rezultate s tistimi iz to"cke (b).

  \end{enumerate}
 
  \item Re"sujemo ena"cbo $y(x)=0.1$, kjer je $y(x)$
  re"sitev za"cetnega problema 
  $$y'(x)=e^{-x^2},\quad y(0)=0.$$

\begin{enumerate}
  \item Napi"si algoritem, ki z Newtonovo metodo ra"cuna 
  pribli"zke re"sitve ena"cbe $y(x)=0.1$.

  \item Izra"cunaj prva dva pribli"zka za re"sitev po Newtonovi 
  metodi, "ce je $x_0=0.2$. Vrednost funkcije $y(x)$ 
  ra"cunaj s trapezno metodo s korakom $h=x_n$ (glej Osnove numeri"cne
  matematike, str. 143).
  
  \item Izra"cunaj prvi pribli"zek pri enakih pogojih kot
  v to"cki (b) "se tako, da vrednost funkcije $y(x)$ 
  izra"cuna"s z metodo Runge-Kutta 4. reda.

  \item[\bf{Dodatek}:] Primerjaj rezultate iz (b) in (c) "ce ve"s, da je 
  to"cna re"sitev ena"cbe $0.100336$.
\end{enumerate}
  

\end{enumerate}


