\
\begin{center}
    {\large IZPIT IZ NUMERI"CNE MATEMATIKE}
    
    8. 6. 2006
\end{center}
\vspace{1.5cm}
\begin{enumerate}
        \item 
	    Re"sujemo ena"cbo
	    \[e^{-x}=\sin(x)\]
	    \begin{enumerate}
		\item Z bisekcijo, dolo"ci pribli"zno lokacijo prvih (najmanj"sih) dveh re"sitev ena"cbe.
		\item Z navadno iteracijo izra"cunaj prvo ni"clo na 
		    3 mesta natan"cno. Na katerem intervalu iteracija konvergira?
		\item Drugo ni"clo poi"s"ci z Newtonovo metodo poi"s"ci na 3 mesta natan"cno.
	    \end{enumerate}
        \item  
	    Podatke za neznano funkcijo $f(x)$
	    \begin{center} 
	    \begin{tabular}{l|ccccc}
		x& $-0.5$ & $0$ & $0.5$ & $1$ & $1.5$.\cr \hline
		f(x)& $1$ & $0.2$ & $0.8$  & $1.8$ & $3$
	    \end{tabular}\\
	    \end{center}
	    aproksimiramo s funkcijo oblike 
	    \[p(x)=a + b x^2.\]
	    \begin{enumerate}
		\item Zapi"si matriko normalnega sistema za dane vrednosti $x$. 
		\item Izra"cunaj 
		    parametra $a$ in $b$ po metodi najmanj"sih kvadratov.
		\item Integral
		    \[ \int_{-0.5}^{1.5}f(x)dx \]
		    izra"cunaj s trapezno in Simpsonovo metodo s korakom $h=1$. Oceni napako trapezne formule,
		    "ce predpostavi"s, da je $|f''(x)|\le 2b$. Rezultata primerjaj z   
		    \[\int_{-0.5}^{1.5}p(x)dx.\]
	    \end{enumerate}
\end{enumerate}
