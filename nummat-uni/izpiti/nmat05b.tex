

\begin{center}
  {\large IZPIT IZ NUMERI"CNE MATEMATIKE\\
    15.6.2005}
\end{center}
\vspace{1.5cm}
\begin{enumerate}
\item Za matriko $A\in\RR^{n\times n}$ poznate njen $LU$ razcep,
        torej $A=L\,U$.
        \begin{itemize}
          \item[a)] Zapi"site {\bf u"cinovit} algoritem za 
            re"sevanje sistema $A^T\,A\,\mathbf{x}=\mathbf{b}$, kjer
            je $\mathbf{b}\in\RR^n$ znani vektor desnih strani sistema.
          \item[b)] Kolik"sna je "casovna zahtevnost algoritma 
            iz to"cke a)?
          \item[c)] Izra"cunajte $\mathbf{x}$, "ce je
            $$
              A=L\,U=
              \left[
                \begin{array}{ccc}
                  1 & 0 & 0\\
                  1 & 1 & 0\\
                  1 & 1 & 1
                \end{array}
               \right]
               \left[
                \begin{array}{ccc}
                  1 & 2 & 3\\
                  0 & -1 & 2\\
                  0 & 0 & -1
                \end{array}
               \right]
            $$
            in $\mathbf{b}=[47,61,192]^T$.
                  \end{itemize}
        
      \item Re"sujete za"cetni problem
        $$y''+\sin{y}=0,\quad y(0)=\frac{\pi}{8},\ y'(0)=0,$$
        ki opisuje nihanje. 
        \begin{enumerate}
        \item Ena"cbo zapi"site kot sistem dveh ena"cb prvega reda.
        \item Izra"cunajte re"sitev sistema z Adams-Bashfortovo  
          metodo drugega reda na intervalu $[0,1]$ s korakom
          $h=0.5$. Dodatno za"cetno vrednost izra"cunajte
          s katerokoli metodo Runge-Kutta drugega reda.
        \item Z interpolacijskim polinomom druge stopnje 
          izra"cunajte pribli"zek za $y'(0.62)$.
        \end{enumerate}
      \end{enumerate}


