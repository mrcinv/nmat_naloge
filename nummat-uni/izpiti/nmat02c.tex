
\begin{center}
  IZPIT IZ NUMERI"CNE MATEMATIKE\\
  6. september 2002
\end{center}

\begin{enumerate}

  \item V ravnini imamo dane tri razli"cne 
    to"cke $\mathbf{T}_i=(x_i,y_i)$, $i=0,1,2$.
    Bezierova parabola na teh treh to"ckah je parametri"cna krivulja,
    definirana takole
    $$\mathbf{B}_2(t):=(1-t)^2\,\mathbf{T}_0+2\,(1-t)\,t\,\mathbf{T}_1+
    t^2\,\mathbf{T}_2,\quad t\in[0,1].
    $$
    Poiskati "zelite to"cko na Bezierovi krivulji, ki je najbli"zje to"cki
    $\mathbf{T}_1$.
    \begin{itemize}
    \item[a)] Zapi"site kvadrat razdalje $f(t):=d^2(\mathbf{T}_1,\mathbf{T})$
      med to"cko $\mathbf{T}_1$ in
      to"cko $\mathbf{T}=(x(t),y(t))$ na krivulji pri parametru $t$.
    \item[b)] Parameter, pri katerem je dose"zena minimalna razdalja, je
      tista re"sitev ena"cbe $f'(t)=0$, ki le"zi na $[0,1]$. 
      Napi"site algoritem, ki z metodo bisekcije poi"s"ce to re"sitev. 
      Za prva dva pribli"zka vzemite kar kraji"s"ci intervala $[0,1]$.
    \item[c)] Zgoraj opisani problem re"site za primer
      $\mathbf{T}_0=(0,0)$, $\mathbf{T}_1=(1,1)$ in $\mathbf{T}_2=(2,0)$.
      Nari"site skico!
    \item[d)] Dodatek: Zakaj re"sitev zgornjega problema vedno obstaja?
    \end{itemize}
    
  \item Re"sujete za"cetni problem
    $$y''(x)+y(x)=0,\quad y(0)=1,\ y'(0)=0.
    $$
    \begin{itemize}
      \item[a)] Poi"s"cite to"cno re"sitev problema.
      \item[b)] Problem na intervalu $[0,1]$ re"site numeri"cno tako, da
        ga prevedete na sistem dveh diferencialnih ena"cb prvega reda in
        uporabite Eulerjevo metodo s korakom $h=0.25$.
      \item[c)] Kolik"sna je globalna napaka v to"cki $x=1$ pri metodi iz b)?
    \end{itemize}
\end{enumerate}



         
