\documentclass[12pt,a4paper]{article}

\usepackage[slovene]{babel}

\def\bfm#1{{\dimen0=.01em\dimen1=.009em\makebold{$#1$}}}
 
\def\makebold#1{\mathord{\setbox0=\hbox{#1}%
       \copy0\kern-\wd0%
       \raise\dimen1\copy0\kern-\wd0%
       {\advance\dimen1 by \dimen1\raise\dimen1\copy0}\kern-\wd0%
       \kern\dimen0\raise\dimen1\copy0\kern-\wd0%
       {\advance\dimen1 by \dimen1\raise\dimen1\copy0}\kern-\wd0%
       \kern\dimen0\raise\dimen1\copy0\kern-\wd0%
       {\advance\dimen1 by \dimen1\raise\dimen1\copy0}\kern-\wd0%
       \kern\dimen0\raise\dimen1\copy0\kern-\wd0%
       \kern\dimen0\box0}}   


\pagestyle{empty}

\begin{document}
\begin{center}
  IZPIT IZ NUMERI"CNE MATEMATIKE\\
  18. junij 2001
\end{center}

\begin{enumerate}

  \item Re"sujemo sistem linearnih ena"cb
    $$\left[
    \begin{array}{cc}
      U & A\\
      0 & L
    \end{array}
    \right]\,\bfm{x}=\bfm{b},$$
    kjer je $U$ zgornje trikotna matrika, $L$ spodnje trikotna matrika in
    $A$ poljubna matrika (vse so reda $n\times n$).


  \begin{enumerate}
   
     \item Napi"site algoritem za re"sevanje tega sistema ena"cb.
       

     \item Pre"stejte "stevilo potrebnih operacij (mno"zenj in deljenj).
       
       

     \item Re"site sistem v primeru, ko je
         $$U=
         \left[
         \begin{array}{ccc}
            1 & 2 & 3\\
            0 & 1 & 2\\
            0 & 0 & 1
         \end{array}
         \right],
         $$
         ter $A=U$, $L=U^T$ in $\bfm{b}=[12,6,2,2,5,8]^T$.       
       
  \end{enumerate}

  \item Znano je, da lahko integrale oblike
    $$\int_{-1}^{1}f(x)\,dx$$
    u"cinkovito ra"cunamo po pribli"zni formuli
    $$\int_{-1}^{1}f(x)\,dx\approx \alpha_1\,f(x_1)+\alpha_2\,f(x_2),$$
    kjer sta $x_1$ in $x_2$ ni"cli polinoma $L_2(x)=3\,x^2/2-1/2$.
    

  \begin{enumerate}
    
    \item Dolo"cite $\alpha_1$ in $\alpha_2$ tako, da bo formula "cim bolj
          natan"cna.

    \item Ugotovite za katere polinome formula ni ve"c to"cna.
      
    \item S pomo"cjo te formule izra"cunajte integral
          $$\int_{-1}^{1}\frac{1}{1+x^2}\,dx$$
          in izra"cunajte kolik"sna je absolutna napaka rezultata.
      
          

  \end{enumerate} 
\end{enumerate}


\end{document}
         
