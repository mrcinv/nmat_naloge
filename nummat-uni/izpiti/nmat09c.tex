

\begin{center}
	IZPIT IZ NUMERI"CNE MATEMATIKE\\
  23. september 2009
\end{center}

\vspace{1cm}

\begin{enumerate}
\item V spodnji tabeli so vrednosti funkcije $\sin$. 
         $$\begin{array}{l|rrrrr}
	      x & 0  & 0.3 & 0.6 & 0.9 \\ \hline
	      \sin x & 0 & 0.295 &   0.564  & 0.783
			\end{array}
				$$
    \begin{enumerate}
			\item S kvadratno in kubično interpolacijo podatkov iz zgornje tabele poišči približno vrednost za $\sin 0.8$.
			\item Kolikšna je absolutna napaka v obeh primerih?
			\item S pomočjo inverzne interpolacije s kvadratnim polinomom poišči rešitev enačbe
				\[x-\sin x=0.08\]
    		in določi relativno napako rešitve, ki jo dobiš.
				\emph{Navodilo:} Sestavi interpolacijsko tabelo za inverzno funkcijo funkcije $f(x)=x-\sin x$.\\ 
		\end{enumerate}
% 2. naloga
\item Integral funkcije na intervalu $[0,h]$
"zelimo izra"cunati po pribli"zni formuli
$$\int_{0}^h f(x)\,dx\approx \alpha\,f(0)+\beta\,f(\gamma).$$ 
  \begin{enumerate}
    \item Dolo"cite $\alpha$, $\beta$ in $\gamma$ tako, da bo formula
    "cim vi"sjega reda.
      
    \item Iz zgornje formule izpeljite sestavljeno pravilo in ga uporabite za izračun integrala
    $$\int_{0}^{\pi/2} \sin{x}\,dx$$
    s korakom $\pi/4$.

    \item Izra"cunajte "se pribli"zek s sestavljenim trapeznim pravilom pri isti
    dol"zini koraka in primerjajte oba pribli"zka s to"cno vrednostjo.
    Kolik"sni sta absolutni napaki?

   \end{enumerate}

\end{enumerate}


