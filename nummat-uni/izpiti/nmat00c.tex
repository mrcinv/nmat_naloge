
\begin{center}
  IZPIT IZ NUMERI"CNIH METOD\\
  15. september 2000
\end{center}
\begin{enumerate}

  \item Funkcija $f$ je predstavljena s tabelo vrednosti
    $$
    \begin{array}{c|ccc|c}
      x&0&1&2&3\\
      \hline
      f(x)&3&2&3&y
    \end{array}
    $$
  \begin{enumerate}
    \item Zapi"si interpolacijski polinom, ki interpolira vrednosti funkcije
    v to"ckah $0$, $1$ in $2$ v Newtonovi obliki.
    \item Izra"cunaj vrednost polinoma iz (a) v to"cki $x=3$.
    \item Dolo"ci $y$ tako, da bo tabela predstavljala kvadratno 
    funkcijo.
  \end{enumerate}

  \item Numeri"cno bi radi izra"cunali vrednost integrala
    $$\int_0^1 e^{\sin{x}}\,dx.$$
  \begin{enumerate}
      \item Izra"cunaj pribli"zno vrednost s trapeznim pravilom pri koraku
      $h=0.25$.
      \item Z istim korakom izra"cunaj pribli"zno vrednost "se s Simpsonovim
        pravilom.
      \item Izra"cunaj absolutno in relativno napako v obeh primerih, "ce ve"s
        da je to"cna vrednost integrala $1.63187$.
  \end{enumerate}

\end{enumerate}


