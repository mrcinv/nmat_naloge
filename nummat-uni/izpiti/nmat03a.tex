\documentclass[12pt,a4paper]{article}

\usepackage[slovene]{babel}

\def\bfm#1{{\dimen0=.01em\dimen1=.009em\makebold{$#1$}}}
 
\def\makebold#1{\mathord{\setbox0=\hbox{#1}%
       \copy0\kern-\wd0%
       \raise\dimen1\copy0\kern-\wd0%
       {\advance\dimen1 by \dimen1\raise\dimen1\copy0}\kern-\wd0%
       \kern\dimen0\raise\dimen1\copy0\kern-\wd0%
       {\advance\dimen1 by \dimen1\raise\dimen1\copy0}\kern-\wd0%
       \kern\dimen0\raise\dimen1\copy0\kern-\wd0%
       {\advance\dimen1 by \dimen1\raise\dimen1\copy0}\kern-\wd0%
       \kern\dimen0\raise\dimen1\copy0\kern-\wd0%
       \kern\dimen0\box0}}   

\usepackage{amsfonts}

%Real and complex numbers
\def\RR{\mathbb{R}}
\def\CC{\mathbb{C}}


\pagestyle{empty}

\begin{document}
\begin{center}
  IZPIT IZ NUMERI"CNE MATEMATIKE\\
  13. junij 2003
\end{center}
\vspace{1cm}

\begin{enumerate}

  \item Naj bo $A\in\RR^{n\times n}$ diagonalno dominantna tridiagonalna
   matrika (od ni"c razli"cni elementi so lahko le na diagonali in na
   obeh stranskih obdiagonalah). Znano je, da v tem primeru Jacobijeva
   metoda za re"sevanje sistema ena"cb $A\bfm{x}=\bfm{b}$
   konvergira za vsak za"cetni pribli"zek.
  \begin{itemize}
     \item[a)] Zapi"site {\bf u"cinkovit algoritem} 
     (upo"stevajte strukturo matrike in ne
     izvajajte nepotrebih operacij) za re"sevanje sistema
     $A\bfm{x}=\bfm{b}$ z Jacobijevo metodo. Pre"stejte "stevilo potrebnih
     operacij (mno"zenj in deljenj) na enem koraku metode.
     \item[b)] Izra"cunajte drugi pribli"zek
     (torej $\bfm{x}^{(2)}$) s to metodo, "ce je
     $$A=\pmatrix{4 & 1 & 0 & 0\cr 1 & 3 & -1 & 0\cr
                0 & 1 & 4 & 2\cr 0 & 0 & 1 & -3},
     \quad \bfm{b}=[6,4,22,-9]^T,\quad
         \bfm{x}^{(0)}=[0,0,0,0]^T.$$
     \item[c)] Kako bi spremenili algoritem iz to"cke a), da bi deloval
     za petdiagonalno matriko (elementi so lahko od ni"c razli"cni le 
     na glavni in dveh stranskih diagonalah)?
  \end{itemize}
 
  \item Tabelo podatkov
    $$\begin{array}{|r|c|c|c|}
        \hline
         x_i & 0 & \pi/4 & \pi/2 \\ \hline
         y_i & 1.1 & 1.5 & 0.9 \\ \hline
      \end{array}
    $$
    aproksimirate s funkcijo $f(x)=A\,\sin{x}+B\,\cos{x}$ po metodi 
    najmanj"sih kvadratov.
    \begin{itemize}
      \item[a)] Zapi"site normalni sistem ena"cb za parametra $A$ in $B$.
      \item[b)] Izra"cunajte parametra $A$ in $B$.
      \item[c)] Izra"cunajte napako metode najman"sih kvadratov, torej
      vsoto 
      $$\sum_{i=1}^3(f(x_i)-y_i)^2.$$

     \end{itemize}
\end{enumerate}
   




\end{document}
         
