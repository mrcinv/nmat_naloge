\
\begin{center}
    {\large IZPIT IZ NUMERI"CNE MATEMATIKE}
    
    10. 6. 2008
\end{center}
\vspace{1.5cm}
\newcommand{\si}{\mathrm{si}}
\begin{enumerate}
    \item 
	Vrednosti funkcije integralski sinus, ki je definirana kot
	\[\si(x)=\int_0^x \frac{\sin(t)}{t}\,dt,\] 
	so podane v tabeli
	\[\begin{array}{c|cccccc}
	x& 1&	1.1&	1.2&	1.3&	1.4&	1.5\cr\hline
	\si(x)&0.94608&   1.02869&   1.10805&   1.18396&   1.25623&1.32468
    \end{array}\]
    Iščemo vrednost \(x_0\), pri kateri je \(\si(x_0)=1\).
    \begin{enumerate}
	\item Z uporabo linerne interpolacije približno izračunaj \(x_0.\) 
	\item Koliko se vrednost \(\si\) v približku razlikuje od 1? 
	    Integral računaj s Simpsonovo formulo.
	\item Približek izboljšaj še z enim korakom Newtonove metode. Uporabiš lahko prej izračunan integral. 
	\item Koliko je sedaj razlika med vrednostjo funkcije \(\si\) in 1.
	    \end{enumerate}
% druga naloga
\item  
    Dan je začetni problem za NDE 2. reda
	\[y''(x)+y(x)=0\] z začetnimi pogoji $y(1)=0.5$ ter $y'(1)=0.1$.
    \begin{enumerate}
	\item
	Izračunaj točno rešitev, če veš, da je oblike
	\[y(x)=A\sin(x)+B\cos(x).\]
    \item Z Eulerjevo metodo s korakom $h=0.1$ poišči vrednost rešitve 
	$y(1.2)$. 
    \item Kolikšna je globalna napaka?
	Kolikšna je lokalna napaka na prvem koraku?
    \item
	\emph{Dodatek:} Kolikšna je  lokalna napaka na drugem koraku?	    
    \end{enumerate}
\end{enumerate}
