\documentclass[12pt,a4paper]{article}

\usepackage[slovene]{babel}
\usepackage{amsfonts}
%Real and complex numbers
\def\RR{\mathbb{R}}
\def\CC{\mathbb{C}}
\def\NN{\mathbb{N}}
% Boldface math
\def\bfm#1{{\dimen0=.01em\dimen1=.009em\makebold{$#1$}}}

\def\makebold#1{\mathord{\setbox0=\hbox{#1}
       \copy0\kern-\wd0
       \raise\dimen1\copy0\kern-\wd0
       {\advance\dimen1 by \dimen1\raise\dimen1\copy0}\kern-\wd0
       \kern\dimen0\raise\dimen1\copy0\kern-\wd0
       {\advance\dimen1 by \dimen1\raise\dimen1\copy0}\kern-\wd0
       \kern\dimen0\raise\dimen1\copy0\kern-\wd0
       {\advance\dimen1 by \dimen1\raise\dimen1\copy0}\kern-\wd0
       \kern\dimen0\raise\dimen1\copy0\kern-\wd0
       \kern\dimen0\box0}}


\pagestyle{empty}

\begin{document}

\begin{center}
  {\large UVOD V NUMERI"CNE METODE\\
    Re"sitve izpita z dne 7.2.2005\\
    }
\end{center}
\vspace{0.5cm}

\begin{enumerate}

  \item Dolo"cite najmanj"so stopnjo polinoma s katerim
    lahko interpoliramo naslednjo tabelo podatkov
    $$\begin{array}{r|rrrrrr}
      x & -2 & -1 & 0 & 1 & 2 & 3\\ \hline
      y & -5 & 1  & 1 & 1 & 7 & 25
      \end{array}.
    $$
    Prepri"cajte se, da ima polinom najni"zje stopnje, ki interpolira
    zgornjo tabelo, eno samo ni"clo na $[-2,3]$. To ni"clo izra"cunajte s
    tangentno metodo na dve decimalni mesti natan"cno.\\
    {\bf Re"sitev}: Najla"zje je, da polinom kar poi"s"cemo, saj ga
    potrebujemo v drugem delu naloge. Uporabimo Newtonovo shemo
    $$\begin{array}{rrrrrrr}
      -2 & -5 &   &   &   &   &\\
         &    & 6 &   &   &   &\\
      -1 &  1 &   & -3&   &   &\\
         &    & 0 &   & 1 &   &\\
       0 &  1 &   & 0 &   & 0 &\\
         &    & 0 &   & 1 &   &0\\
       1 &  1 &   & 3 &   & 0 &\\
         &    & 6 &   & 1 &   &\\
       2 &  7 &   & 6 &   &   &\\
         &    & 18&   &   &   &\\
       3 & 25 &   &   &   &   &
      \end{array}
    $$
    in dobimo polinom 
    $$p(x)=-5+6(x+2)-3(x+2)(x+1)+(x+2)(x+1)x=
    x^3-x+1.$$ 
    Torej tabelo lahko interpoliramo s polinomom tretje stopnje.\\
    Ker je $p(-2)<0$ in $p(3)>0$, ima $p$ na $[-2,3]$ eno ali tri ni"cle.
    Toda $p'(x)=3x^2-1$, kar pomeni, da sta lokalna ekstrema lahko le
    v $\pm 1/\sqrt{3}$. Hitro preverimo, da je $p''(1/\sqrt{3})>0$ in
    $p(1/\sqrt{3})>0$, torej ima $p$ pozitivno vrednost v lokalnem
    minimumu. Iz tega lahko sklepamo, da je na $[-2,3]$ ena sama ni"cla.\\
    Tangentna metoda za iskanje ni"cle je
    $$x_{n+1}=x_n-\frac{p(x_n)}{p'(x_n)}=x_n-\frac{x_n^3-x_n+1}{3x_n^2-1}=
    \frac{2x_n^3-1}{3x_n^2-1}.$$
    Z za"cetnim pribli"zkom $x_0=-2$ po "stirih korakih dobimo ni"clo $-1.32$
    na dve decimalni mesti natan"cno.

  \item V naslednji tabeli so podatki hitrosti avtomobila ($v$[km/h])
    v prvih dveh sekundah vo"znje
    $$\begin{array}{l|rrrrr}
    v & 40 & 46 & 51 & 55 & 59\\ \hline
    t & 0  & 0.5 & 1 & 1.5 & 2
    \end{array}.$$
    S trapeznim in Simpsonovim pravilom izra"cunajte dol"zino poti 
    (v me\-trih), ki jo je prevozil avtomobil v prvih dveh sekundah. Nato 
    z vrednostmi hitrosti ob "casih $0.5s$, $1s$ in $1.5s$ "cim bolje ocenite
    pospe"sek avtomobila po eni sekundi.\\
    {\bf Re"sitev}: Korak v tabeli je o"citno $h=0.5$. S trapeznim pravilom
    dobimo oceno za pot
    $$s=\frac{0.5}{2}(40+2*46+2*51+2*55+59)/3.6=27.9861m,$$
    s Simpsonovim pa
    $$s=\frac{0.5}{3}(40+4*46+2*51+4*55+59)/3.6=28.0093m$$
    (fizikom menda ni treba posebej razlagati, 
    zakaj faktor $3.6$ v imenovalcu!).
    Pospe"sek najbolje ocenimo s sredinsko formulo za odvod
    $$v'(t)\approx \frac{v(t+h)-v(t-h)}{2h}.$$
    V na"sem primeru dobimo $v'(1)\approx 2.5m/s^2$.
  \item Naj bo 
    $$A=\left(
      \begin{array}{ccccccc}
        x & 0 & 0 & \cdots & 0 & 0 & a_0\\
        -1& x & 0 & \cdots & 0 & 0 & a_1\\
        0 & -1& x & \cdots & 0 & 0 & a_2\\
        \vdots & \vdots & \ddots & \ddots & \vdots & \vdots & \vdots\\
        0 & 0 & \cdots & \cdots & -1 & x & a_{n-1}\\
        0 & 0 & \cdots & \cdots & 0 & -1 & x
      \end{array}\right) .
    $$
      Izra"cunajte determinanto matrike $A$ tako, da najprej
      izra"cunate njen $LU$ razcep in upo"stevate dejstvo
      $\det A=\det U=\prod_{i=1}^{n+1}u_{ii}$.\\
    {\bf Re"sitev}: Dokaj preprosto je izra"cunati $LU$ razcep
    matrike $A$, saj je potrebno izra"cunati kvociente v $L$
    samo na spodnji obdiagonali, v $U$ pa le elemente v zadnjem
    stolpcu. Dobimo
    $$L=\left(
      \begin{array}{rrrrr}
        1 & 0 & 0 & \cdots & 0\\
        -\frac{1}{x} & 1 & 0 & \cdots &0\\
        0 & -\frac{1}{x} & 1 & \cdots &0\\
        \vdots & \vdots&\ddots&\ddots&\vdots\\
        0&0&\cdots&-\frac{1}{x}&1
      \end{array}\right)
  $$
  in
  $$
      U= \left(
      \begin{array}{rrrr}
        x & 0  & \cdots &a_0\\
        0 & x  & \cdots &a_1+\frac{a_0}{x}\\
        \vdots  & \vdots & \ddots &\vdots\\
        0 & 0 &\cdots &x+\frac{a_{n-1}}{x}+\cdots+\frac{a_0}{x^n}
      \end{array}\right).
   $$
   Torej je 
   \begin{eqnarray*}
     \det A=\det U&=&x^n\left(x+\frac{a_{n-1}}{x}+\cdots
     +\frac{a_0}{x^n}\right)\\
     &=&x^{n+1}+a_{n-1}x^{n-1}+\cdots +a_1 x+a_0.
   \end{eqnarray*}
  \item Nekdo je re"seval diferencialno ena"cbo $y''(x)=-x\,y(x)$ 
    na intervalu $[0,0.2]$ z Eulerjevo metodo s korakom $h=0.1$.
    Za vrednost funkcije $y(0.2)$ je dobil numeri"cni pribli"zek $0.2$,
    za odvod $y'(0.2)$ pa $0.999$. Iz\-ra\-"cu\-najte vrednosti 
    $y(0)$ in $y'(0)$?\\
    {\bf Re"sitev}: Ena"cbo zapi"semo kot sistem prvega reda
    $$\bfm{Y}'=\bfm{F}(x,\bfm{Y})=\left(
    \begin{array}{r}
      y_2\\
      -x\,y_1
    \end{array}\right),\quad \bfm{Y}=
    \left(
    \begin{array}{r}
      y_1\\
      y_2
    \end{array}\right)=
    \left(
    \begin{array}{r}
      y\\
      y'
    \end{array}\right).
    $$
    Zapi"simo dva koraka Eulerjeve metode
    \begin{eqnarray*}
      \bfm{Y}_1&=&\bfm{Y}_0+h\bfm{F}(x_0,\bfm{Y}_0)=
    \left(
    \begin{array}{r}
      y_1(0)\\
      y_2(0)
    \end{array}\right)+
    h
    \left(
    \begin{array}{r}
      y_2(0)\\
      -0\,y_1(0)
    \end{array}\right)\\
    &=&
    \left(
    \begin{array}{r}
      y_1(0)+h\,y_2(0)\\
      y_2(0)
    \end{array}\right),
    \end{eqnarray*}
    \begin{eqnarray*}
      \bfm{Y}_2&=&\bfm{Y}_1+h\bfm{F}(x_1,\bfm{Y}_1)=
    \left(
    \begin{array}{r}
      y_1(0)+h\,y_2(0)\\
      y_2(0)
    \end{array}\right)\\
    &+&h
    \left(
    \begin{array}{r}
      y_2(0)\\
      -0.1\,(y_1(0)+h\,y_2(0))
    \end{array}\right)
    =
    \left(
    \begin{array}{r}
      0.2\\
      0.999
    \end{array}\right),
    \end{eqnarray*}
Upo"stevamo, da je $h=0.1$ in dobimo sistem ena"cb
\begin{eqnarray*}
   y_1(0)+0.2\,y_2(0)&=&0.2\\
   -0.01\,y_1(0)+0.999\,y_2(0)&=&0.999,
\end{eqnarray*}
katerega re"sitev je $y_1(0)=0$ in $y_2(0)=1$.
\end{enumerate}
\end{document} 
