\documentclass[12pt,a4paper]{article}

\usepackage[slovene]{babel}
\usepackage{amsfonts,amsmath}
%Real and complex numbers
\def\RR{\mathbb{R}}
\def\CC{\mathbb{C}}
\def\NN{\mathbb{N}}
% Boldface math
\def\bfm#1{{\dimen0=.01em\dimen1=.009em\makebold{$#1$}}}

\def\makebold#1{\mathord{\setbox0=\hbox{#1}
       \copy0\kern-\wd0
       \raise\dimen1\copy0\kern-\wd0
       {\advance\dimen1 by \dimen1\raise\dimen1\copy0}\kern-\wd0
       \kern\dimen0\raise\dimen1\copy0\kern-\wd0
       {\advance\dimen1 by \dimen1\raise\dimen1\copy0}\kern-\wd0
       \kern\dimen0\raise\dimen1\copy0\kern-\wd0
       {\advance\dimen1 by \dimen1\raise\dimen1\copy0}\kern-\wd0
       \kern\dimen0\raise\dimen1\copy0\kern-\wd0
       \kern\dimen0\box0}}


\pagestyle{empty}

\begin{document}

\begin{center}
  {\large UVOD V NUMERI"CNE METODE\\
    2. izpit, 5.5.2005\\
    }
\end{center}
\vspace{1cm}

\begin{enumerate}

  \item S tangentno metodo ra"cunate najmanj"so ni"clo
      {\sl Laguerreovega} polinoma $L_3$. Naj bo za"cetni pribli"zek
      $x_0=0$. Zapi"site pribli"zek $x_2$ na $4$ decimalna mesta.
      Laguerreovi polinomi so dani rekurzivno takole
      \begin{align*}
        L_0(x)&=1,\\
        L_1(x)&=1-x,\\
        (n+1)\,L_{n+1}(x)&=(2\,n+1-x)\,L_n(x)-n\,L_{n-1}(x).
      \end{align*}
      Dolo"cite red konvergence metode pri iskanju te ni"cle.\\
      {\sl Namig: Najprej izra"cunajte koeficiente polinoma $L_3$.}
  
  \item {\sl Gauss--Chebysheva} kavadraturna formula
      se glasi
      $$\int_{-1}^{1}\frac{f(x)}{\sqrt{1-x^2}}\,dx=
        \sum_{k=1}^n \alpha_k\,f\left(\cos\left(\frac{2k-1}{2n}\pi\right)\right)+
        \frac{2\,\pi}{2^{2n}(2\,n)!}f^{(2n)}(\xi),$$ 
      kjer je $\xi\in[-1,1]$. 
      Dolo"cite konstanti $\alpha_1$ in $\alpha_2$ ter 
      izra"cunajte numeri"cni pribli"zek
      za integral
      $$\int_{-1}^{1}\frac{e^{x}}{\sqrt{1-x^2}}\,dx$$
      v primeru $n=2$. S pomo"cjo zgoraj zapisane formule
      ocenite absolutno napako pri izra"cunu.

  \item Tabelo podatkov $(x_i,f_i)$, $i=1,2,\dots,n$, 
      aproksimirate s konstanto po metodi
      najmanj"sih kvadratov. Preverite, da je konstanta enaka povpre"cni vrednosti
      "stevil $f_i$.  

  \item Nihanje matemati"cnega nihala opi"semo z eksaktnim  
    $$\ddot\varphi(t)+\frac{g}{\ell}\sin(\varphi(t))=0, \quad \varphi(0)=\varphi_0,\ 
      \dot\varphi(0)=\dot\varphi_0$$
    in poenostavljenim 
     $$\ddot\varphi(t)+\frac{g}{\ell}\varphi(t)=0, \quad \varphi(0)=\varphi_0,\ 
      \dot\varphi(0)=\dot\varphi_0$$
    za"cetnim problemom. 
    Izra"cunajte absolutno vrednost razlike kotnih hitrosti, 
    ki ju dobimo z re"sevanjem prvega in drugega
    problema z Eulerjevo metodo po $0.5s$. Korak naj bo $0.25s$, $\ell=10m$,
    $\varphi_0=0.5$ in $\dot\varphi_0=0s^{-1}$. Za te"zni pospe"sek vzemite
    $g=9.8m/s^2$.

\end{enumerate}
\end{document} 
