
\begin{center}
  IZPIT IZ NUMERI"CNE MATEMATIKE\\
  30. junij 2003
\end{center}
\vspace{1cm}

\begin{enumerate}

  \item Funkcija $f$ je podana s predpisom takole
  $$f(x)=
  \left\{
  \begin{array}{rl}
    -x^2+2, & -1\leq x\leq 1\\
    (1-x)(x-2)x+1,&1\leq x\leq 3\\
    -x-2, & x\geq 3.
  \end{array}
  \right.
  $$  

  \begin{itemize}
    \item[a)] Prepri"cajte se, da z metodo regula falsi gotovo
    poi"s"cemo ni"clo funkcije na intervalu $[-1,5]$.
    \item[b)] Ni"clo poi"s"cite z metodo regula falsi na eno 
    decimalno mesto natan\-"cno.
    \item[c)] Koliko korakov bisekcije je potrebnih, da ni"clo poi"s"cete
    na 8 mest natan"cno, "ce je za"cetni interval $[-1,5]$?
  \end{itemize}
 
  \item Re"sujete za"cetni problem
  $$y''(x)+y(x)=0,\quad y(0)=1,\ y'(0)=0$$
  na intervalu $[0,0.5]$.

    \begin{itemize}
      \item[a)] Prevedite problem na re"sevanje sistema dveh 
      diferencialnih ena"cb prvega reda.

      \item[b)] Re"site sistem ena"cb z Eulerjevo metodo s korakom $h=0.1$.

      \item[c)] Izra"cunajte globalno napako numeri"cne re"sitve za"cetnega 
      problema v to"cki $x=0.5$.

     \end{itemize}
\end{enumerate}
   




