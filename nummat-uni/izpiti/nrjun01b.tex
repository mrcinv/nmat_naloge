\documentclass[12pt,a4paper]{article}

\usepackage[slovene]{babel}

\def\bfm#1{{\dimen0=.01em\dimen1=.009em\makebold{$#1$}}}
 
\def\makebold#1{\mathord{\setbox0=\hbox{#1}%
       \copy0\kern-\wd0%
       \raise\dimen1\copy0\kern-\wd0%
       {\advance\dimen1 by \dimen1\raise\dimen1\copy0}\kern-\wd0%
       \kern\dimen0\raise\dimen1\copy0\kern-\wd0%
       {\advance\dimen1 by \dimen1\raise\dimen1\copy0}\kern-\wd0%
       \kern\dimen0\raise\dimen1\copy0\kern-\wd0%
       {\advance\dimen1 by \dimen1\raise\dimen1\copy0}\kern-\wd0%
       \kern\dimen0\raise\dimen1\copy0\kern-\wd0%
       \kern\dimen0\box0}}   


\pagestyle{empty}

\begin{document}
\begin{center}
  IZPIT IZ NUMERI"CNE MATEMATIKE\\
  27. junij 2001
\end{center}

\begin{enumerate}

  \item Tabelo podatkov
   $$\begin{array}{|c||c|c|}
        \hline
        x & 1 & 2\\ \hline
        y & 1 & 2\\ \hline
      \end{array}
   $$
   "zelite aproksimirati po metodi najmanj"sih kvadratov s funkcijo
   oblike 
   $$y(x)=e^{\lambda\,x}.$$ 
    
  \begin{enumerate}
   
     \item Napi"site funkcijo, katere minimum je potrebno poiskati.
  
     \item Zapi"site ena"cbo, ki dolo"ca optimalni $\lambda$.
       
     \item Numeri"cno izra"cunajte optimalni $\lambda$ z Newtonovo metodo
           na tri decimalna mesta. Ustrezni za"cetni pribli"zek dolo"cite
           sami. 
        
  \end{enumerate}

  \item Re"sujete za"cetni problem
  $$y'(x)=-20\,y(x)+20,\quad y(0)=1.01.$$
    
  \begin{enumerate}
    
    \item Re"site problem z Eulerjevo metodo na intervalu $[0,1]$ s 
          korakom $h=0.25$.
          
    \item Poi"s"cite to"cno re"sitev za"cetnega problema.
      
    \item Kolik"sna je lokalna napaka v to"cki $x=1/2$ in kolik"sna
          globalna napaka pri $x=1$. Zakaj tako?
            
  \end{enumerate} 
\end{enumerate}


\end{document}
         
