\documentclass[12pt,a4paper]{article}

\usepackage[latin2]{inputenc}
\usepackage[slovene]{babel}
\usepackage{amsfonts}
% Mno"zice realnih, naravnih in kompleksnih "stevil
\def\RR{\mathbb{R}}
\def\NN{\mathbb{N}}
\def\CC{\mathbb{C}}
\addtolength{\textheight}{4cm}
\addtolength{\voffset}{-2cm}

\newtheorem{definicija}{Definicija}
\newtheorem{primer}{Primer}
% Odebeljene "crke v math na"cinu 
% (povezto po L.L. Schumaker - makro
% za St. Malo Proceedings
\def\bfm#1{{\dimen0=.01em\dimen1=.009em\makebold{$#1$}}}

\def\makebold#1{\mathord{\setbox0=\hbox{#1}%
       \copy0\kern-\wd0%
       \raise\dimen1\copy0\kern-\wd0%
       {\advance\dimen1 by \dimen1\raise\dimen1\copy0}\kern-\wd0%
       \kern\dimen0\raise\dimen1\copy0\kern-\wd0%
       {\advance\dimen1 by \dimen1\raise\dimen1\copy0}\kern-\wd0%
       \kern\dimen0\raise\dimen1\copy0\kern-\wd0%
       {\advance\dimen1 by \dimen1\raise\dimen1\copy0}\kern-\wd0%
       \kern\dimen0\raise\dimen1\copy0\kern-\wd0%
       \kern\dimen0\box0}}
\pagestyle{empty}

\begin{document}

\begin{center}
  {\large IZPIT IZ NUMERI"CNE MATEMATIKE\\
    14. 9. 2005}
\end{center}
\vspace{1.5cm}
\begin{enumerate}
  \item Dan je sistem linearnih ena�b
      \begin{eqnarray*}
	  0.1410\cdot 10^{-2}x+0.4004\cdot 10^{-1}y&=0.1142\cdot 10^{-1}\\
	  0.2000\cdot 10^0x+0.4912\cdot 10^1y&=0.1428\cdot 10^1
      \end{eqnarray*}
  \begin{enumerate}
      \item Izra�unajte re�itev sistema brez pivotiranja!
      \item Izra�unajte re�itev sistema z delnim pivotiranjem!
      \item Izra�unajte �tevilo ob�utljivosti matrike sistema v neskon�ni normi! 
  \end{enumerate}
  Pri prvih dveh vpra�anjih ra�unajte na $5$ decimalnih mest.
  \item Nekdo je re"seval diferencialno ena"cbo $y''(x)=-2\,y'(x)+3\,y(x)$ 
    na intervalu $[0,0.2]$ z Eulerjevo metodo s korakom $h=0.1$.
    Za vrednost funkcije $y(0.2)$ je dobil numeri"cni pribli"zek $0.2$,
    za odvod $y'(0.2)$ pa $0.999$. 
    \begin{enumerate}    
	\item Iz\-ra\-"cu\-najte vrednosti  $y(0)$ in $y'(0)$! 
	\item Poi��ite splo�no re�itev diferencialne ena�be!
	\item Koliko je to�na vrednost $y(0.2)$? 
    \end{enumerate}
    \end{enumerate}

\end{document}
