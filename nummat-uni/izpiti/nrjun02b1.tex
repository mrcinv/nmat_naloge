\documentclass[12pt,a4paper]{article}
\addtolength{\textwidth}{4cm}
\addtolength{\hoffset}{-2cm}
\addtolength{\voffset}{-3cm}
\addtolength{\textheight}{3cm}
\usepackage[slovene]{babel}

\def\bfm#1{{\dimen0=.01em\dimen1=.009em\makebold{$#1$}}}
 
\def\makebold#1{\mathord{\setbox0=\hbox{#1}%
       \copy0\kern-\wd0%
       \raise\dimen1\copy0\kern-\wd0%
       {\advance\dimen1 by \dimen1\raise\dimen1\copy0}\kern-\wd0%
       \kern\dimen0\raise\dimen1\copy0\kern-\wd0%
       {\advance\dimen1 by \dimen1\raise\dimen1\copy0}\kern-\wd0%
       \kern\dimen0\raise\dimen1\copy0\kern-\wd0%
       {\advance\dimen1 by \dimen1\raise\dimen1\copy0}\kern-\wd0%
       \kern\dimen0\raise\dimen1\copy0\kern-\wd0%
       \kern\dimen0\box0}}   


\pagestyle{empty}

\begin{document}
{\large \begin{center}
  IZPIT IZ NUMERI"CNE MATEMATIKE\\
  26. junij 2002
\end{center}

\begin{enumerate}

  \item Za realno matriko $A_n$ reda $n\times n$ velja
    \begin{eqnarray*}
      &&a_{i,n-i+1}\neq 0, \quad i=1,2,\dots,n,\\
      &&a_{i,j}=0, \quad j>n-i+1.
    \end{eqnarray*}
    Ostali elementi so poljubni.
    \begin{itemize}
      \item[a)] Napi"site en primer take matrike $A_4$.
      \item[b)] Zapi"site u"cinkovit algoritem za re"sevanje sistema
        $A_n\,\bfm{x}=\bfm{b}$, kjer je $\bfm{b}$ dani vektor dimenzije $n$.
        Kolik"sna je "casovna zahtevnost va"sega postopka?
      \item[c)] Kako bi popravili algoritem iz to"cke b), da bi deloval
        tudi na sistemu $A_n^T\,\bfm{x}=\bfm{b}$?
      \item[d)] Naj bo $n=4$. Re"site sistem $A_4\,\bfm{x}=[20,10,4,1]^T$, 
        "ce je
        $$a_{i,j}=\max\{6-i-j,0\},\quad i,j=1,2,\dots,4.$$
    \end{itemize}

  \item Integral ute"zene funkcije $f(x)\sqrt{x}$ ra"cunamo pribli"zno
    po formuli
    $$\int_{a}^{a+h} f(x)\sqrt{x}\,dx\approx \alpha f(a)+\beta f(a+h).$$
    \begin{itemize}
      \item[a)] Dolo"cite parametra $\alpha$, $\beta$ tako, da bo
      formula "cim vi"sjega reda. (Nasvet: lahko predpostavite, da je $a=0$.)
      \item[b)] Napi"site algoritem za ra"cunanje integrala
        $$\int_{a}^b f(x)\sqrt{x}\,dx$$
        s sestavljenim pravilom iz to"cke a).
      \item[c)] Pravilo iz to"cke b) uporabite pri izra"cunu integrala
        $$\int_0^{0.2}e^{-\sqrt{x}}\sqrt{x}\,dx$$
        s korakom $h=0.1$.
        Kolik"sna je napaka rezultata, "ce veste, da je
        $$\int e^{-\sqrt{x}}\sqrt{x}\,dx=
        -2\,e^{-\sqrt{x}}(2+2\sqrt{x}+x)+C?$$
     \end{itemize}
\end{enumerate}

}
\end{document}
         
