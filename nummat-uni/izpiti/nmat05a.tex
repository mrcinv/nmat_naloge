\documentclass[12pt,a4paper]{article}
\usepackage[slovene]{babel}
\usepackage{amsfonts,graphicx}
% Mno"zice realnih, naravnih in kompleksnih "stevil
\def\RR{\mathbb{R}}
\def\NN{\mathbb{N}}
\def\CC{\mathbb{C}}
\addtolength{\textheight}{4cm}
\addtolength{\voffset}{-2cm}

\newtheorem{definicija}{Definicija}
\newtheorem{primer}{Primer}
% Odebeljene "crke v math na"cinu 
% (povezto po L.L. Schumaker - makro
% za St. Malo Proceedings
\def\bfm#1{{\dimen0=.01em\dimen1=.009em\makebold{$#1$}}}

\def\makebold#1{\mathord{\setbox0=\hbox{#1}%
       \copy0\kern-\wd0%
       \raise\dimen1\copy0\kern-\wd0%
       {\advance\dimen1 by \dimen1\raise\dimen1\copy0}\kern-\wd0%
       \kern\dimen0\raise\dimen1\copy0\kern-\wd0%
       {\advance\dimen1 by \dimen1\raise\dimen1\copy0}\kern-\wd0%
       \kern\dimen0\raise\dimen1\copy0\kern-\wd0%
       {\advance\dimen1 by \dimen1\raise\dimen1\copy0}\kern-\wd0%
       \kern\dimen0\raise\dimen1\copy0\kern-\wd0%
       \kern\dimen0\box0}}
\pagestyle{empty}

\begin{document}

\begin{center}
  IZPIT IZ NUMERI"CNE MATEMATIKE\\
  6. junij 2005
\end{center}
\vspace{1cm}

\begin{enumerate}
  \item Temperatura v razgretem avtomobilu se spreminja s 
    "casom pribli"zno takole
    $$T(t)=T_0+\Delta T_0\,\exp(k\,t),$$
    kjer je $T(t)$ temperatura ob
    "casu $t$ v notranjosti avtomobila, $T_0$ znana konstantna zunanja
    temperatura, $\Delta T_0$ razlika med notranjo in zunanjo
    temperaturo ob za"cetku hlajenja (torej ob "casu $t=0$) in $k$
    parameter, ki opisuje hitrost hlajenja.
  \begin{itemize}
    \item[a)] Zapi"site algoritem, ki bo ocenil $\Delta T_0$ in $k$
      po metodi najmanj"sih kvadaratov, "ce so dani podatki o 
      notranji temperaturi $T(t_i)$ ob nekaj izbranih "casih $t_i$.\\
      {\sl Namig: logaritmirajte $T(t)-T_0$.}
    \item[b)] Zunanja temperatura je $T_0=30C$. Temperaturo v 
    avtomobilu smo izmerili v minutnih intervalih in dobili naslednjo
    tabelo:%T0=30;k=-0.15/min;dT_0=15;
    $$
      \begin{array}{c|ccccc}
        t & 1 & 2 & 3 & 4 & 5\\ \hline
        T & 42.9106 & 41.1123 & 39.5644 & 38.2322 & 37.0855
      \end{array}.
    $$
    Dolo"cite $\Delta T_0$ in $k$ za ta primer.
    \item[c)] Kolik"sna bo temperatura v to"cki b) po $10$ minutah?
      Po kolik"snem "casu bo temperatura padla na $35C$?
  \end{itemize}
  \item 
    \begin{itemize}
      Integral $I=\int_{a}^{a+h} f(x)\,dx$ ra"cunamo z enostavnim
      trapeznim pravilom s korakom $h$, torej
      $$ I\approx T(h):=\frac{h}{2}\left(f(a)+f(a+h)\right).$$
     \item[a)] Kako bi izra"cunali $h$, "ce poznate $T(h)$?
     \item[b)] Naj bo $a=0$, $f(x)=\exp(-x)$ in $T(h)=0.4016$.
       Izra"cunajte $h$ na tri decimalna mesta.
     \item[c)] Izra"cunajte vrednost integrala $I$ iz to"cke b) "se s
       sestavljenim tra\-pez\-nim pravilom s korakom $h/2$ (kjer ste
       $h$ 
       izra"cunali v to"cki b)) in z Rombergovo metodo izra"cunajte
       bolj"si pribli"zek.
    \end{itemize}
\end{enumerate}


\end{document}
