\documentclass[12pt,a4paper]{article}

\usepackage[slovene]{babel}

\pagestyle{empty}

\begin{document}
\begin{center}
  IZPITI IZ NUMERI"CNIH METOD\\
  jesen 2001
\end{center}
\begin{enumerate}
 \item Re"sujemo sistem linearnih ena"cb
    \begin{eqnarray*}
      x_1+x_2 &=& 1\\
      x_2+x_3 &=& 2\\
              &\vdots&\\
      x_n+x_1 &=& n,
    \end{eqnarray*}
    kjer je $n$ liho "stevilo.
 \begin{itemize}
    \item[a)] Zapi"site sistem v matri"cni obliki in doka"zite,
      da je vedno re"sljiv.
    \item[b)] Predlagajte algoritem za re"sevanje tega sistema in
      pre"stejte "stevilo potrebnih operacij (mno"zenj in deljenj).
      Algoritem naj bo kar se da u"cinkovit.
    \item[c)] Re"site sistem v primeru $n=5$.
 \end{itemize}

 \item Radi bi izra"cunali nekaj decimalk "stevila 
        $\sqrt{2}$. V ta namen re"sujemo ena"cbo $x^2-2=0$
        z navadno iteracijo takole
        $$x_{n+1}=x_n+c\,(x_n^2-2).$$
 \begin{itemize}
      \item[a)] Izra"cunajte prve "stiri pribli"zke, "ce je
        $c=-0.3$ in $x_0=1$.
      \item[b)] Za katere vrednosti parametra $c$ metoda 
        konvergira, "ce vzamemo $x_0$ dovolj blizu $\sqrt{2}$?
      \item[c)] Dolo"cite parameter $c$, pri katerem metoda 
        najhitreje konvergira.
 \end{itemize}

 \item Funkcija $f$ je predstavljena s tabelo vrednosti
    $$
    \begin{array}{c|cccc}
      x&-1&0&2&a\\
      \hline
      f(x)&-4&-2&8&0
    \end{array}
    $$
  \begin{itemize}
     \item[a)] Z metodo {\sl regula falsi} dolo"cite realno "stevilo
      $a$ na dve decimalni mesti tako, da bo tabela predstavljala
      vrednosti funkcije, ki jih interpolira polinom
      $p(x)=x^3+x-2$.
     \item[b)] Kak"sen pa mora biti $a$, da bo 
      $$f[-1,0,2,a]=0?$$
     \item[c)] Naj bo $a=1$. Dolo"cite premico, ki po metodi
      najmanj"sih kvadratov aproksimira podatke iz tabele.
  \end{itemize}

  \item V kraji"s"cih intervala $[a,a+h]$ poznamo vrednosti 
    funkcije $f$ in njenih odvodov. Izpeljati "zelimo 
    integracijsko formulo oblike
    $$\int_{a}^{a+h} f(x)\,dx\approx \alpha\,f(a)+
    \beta\,f'(a)+\gamma\,f(a+h)+\delta\,f'(a+h).$$
  \begin{itemize}
    \item[a)] Dolo"cite koeficiente $\alpha$, $\beta$, $\gamma$
      in $\delta$ tako, da bo formula "cim vi"sjega reda (namig:
      predpostavite lahko, da je $a=0$).
    \item[b)] Na podlagi te formule izpeljite "se 
      sestavljeno integracijsko pravilo.
    \item[c)] S pravilom iz to"cke b) izra"cunajte integral
      $$\int_{0}^{0.3} \sin{x}\,dx,$$
      pri koraku $h=0.1$. Kolik"sna je absolutna napaka rezultata?
  \end{itemize}

  \item Re"sujete za"cetni problem
    $$y''(x)+3\,y'(x)-4\,y(x)=0,\quad y(0)=0,\ y'(0)=5.$$
  \begin{itemize}
     \item[a)] Poi"s"cite to"cno re"sitev problema.
     \item[b)] Prevedite diferencialno ena"cbo na sistem dveh
       diferencialnih ena"cb prvega reda.
     \item[c)] Sistem iz to"cke b) re"site z Eulerjevo metodo
       na intevalu $[0,1]$ s korakom $h=0.25$. Kolik"sna je
       globalna napaka v to"cki $x=1$?
     \end{itemize}
  

\end{enumerate}


\end{document}
         