

\begin{center}
  {\large UVOD V NUMERI"CNE METODE\\
    1. kolokvij, 16.12.2004\\
    }
\end{center}
\vspace{2cm}

\begin{enumerate}

  \item Funkcijo $f(x)=\ln{x}$ imate tabelirano v 
    to"ckah $x_0=10$, $x_1=11$ in $x_2=12$ (vrednosti 
    $f(x_i)$ izra"cunajte sami). Ocenite vrednost $f(11.1)$
    z vrednostjo kvadratnega interpolacijskega polinoma 
    na to"ckah $x_i$, $i=0,1,2$. Nato "cim bolje ocenite napako, ki jo
    pri tem naredite (pri tem ne smete uporabiti to"cne
    vrednosti $f(11.1)$).
    
  \item Izpeljite integracijsko formulo oblike
    $$\int_{0}^1 f(x)\,dx=\alpha\,f(0)+\beta\,f(1)+\gamma\,f'(0)+R(f),$$
    da bo natan"cna za polinome "cim vi"sje stopnje. Nato 
    z njo ocenite integral
    $$\int_{0}^{0.2}\sin{x}\,dx$$
    ter za ta primer izra"cunajte napako $R(f)$.
   
  \item Ni"clo funkcije $f_a(x)=x\,\ln{x}-a$ i"s"cete z navadno iteracijo
   oblike
   $$x_{n+1}=\frac{x_{n}+a}{\ln{x_n}+1}.$$
   Preverite, da je ta iteracija ravno tangentna metoda  za re"sevanje
   ena"cbe $f_a(x)=0$ z redom konvergence vsaj $2$.\\
   Grafi"cno (ali kako druga"ce) utemeljite, da ima $f_1$ eno samo
   ni"clo in jo z zgornjo iteracijo dolo"cite na tri decimalna mesta 
   natan"cno.
  
  \item Naj bo $n\in\NN$, $\alpha\in\RR$ in 
    $$A=\left(
      \begin{array}{cc}
        1 & \alpha \\
        0 & 1
      \end{array}\right).
    $$
    Izra"cunajte $\|A^n\|_p$ za $p=1,2,\infty$.


\end{enumerate}
