

\begin{center}
  IZPIT IZ NUMERI"CNE MATEMATIKE\\
  16. junij 2004
\end{center}
\vspace{1cm}

\begin{enumerate}
  \item Znano je, da vsoto potenc naravnih "stevil $\sum_{i=1}^n i^k$
    lahko izrazimo s polinomom 
    $$p_{k+1}(n)=c_0+c_1(n-1)+c_2(n-1)(n-2)+
    \cdots+c_{k+1}(n-1)\cdots(n-k-1).$$
    Koeficiente polinoma $p_{k+1}$ dobimo z re"sevanjem sistema linearnih
   ena"cb
    $$p_{k+1}(j)=\sum_{i=1}^{j}i^k,\quad j=1,2,\dots,k+2,$$
    torej $A\bfm{c}=\bfm{d}$,
    kjer je $A\in\RR^{(k+2)\times(k+2)}$, 
    $\bfm{c}=(c_0,c_1,\dots,c_{k+1})^T$ in $\bfm{d}=(d_1,d_2,\dots,d_{k+2})^T$.
  \begin{itemize}
    \item[a)] Zapi"site matriko sistema $A$ in desno stran $\bfm{d}$.
    \item[b)] Glede na obliko matrike $A$ predlagajte algoritem za 
      re"sevanje sistema $A\bfm{c}=\bfm{d}$.
    \item[c)] Izra"cunajte $p_2$ z algoritmom iz to"cke b.
  \end{itemize}
  \item Pri izpeljavi formule za ra"cunanje dolo"cenega integrala funkcije
    $$\int_{a}^{a+h}f(x)\,dx\approx \frac{h}{5}\left(2f(a)+3f(a+2h/3)\right)$$
    smo naredili napako pri izra"cunu koeficientov. 
    Formulo smo "zeleli izpeljati tako, da bo
    natan"cna za polinome "cim vi"sje stopnje.
    \begin{itemize}
     \item[a)] Popravite napako v formuli.
     \item[b)] Zapi"site algoritem za izra"cun integrala s sestavljenim
       pravilom na podlagi popravljene formule.
     \item[c)] Izra"cunajte napako pri ra"cunanju integrala funkcije
       $f(x)=\sin{x}$ na $[0,0.2]$ s popravljeno formulo in korakom $h=0.2$.
    \end{itemize}
\end{enumerate}
