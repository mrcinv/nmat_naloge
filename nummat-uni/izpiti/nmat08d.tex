\begin{center}
    {\large IZPIT IZ NUMERI"CNE MATEMATIKE}
    
    11. 9. 2008
\end{center}
\vspace{1.5cm}
\begin{enumerate}
        \item 
	    Re"sujemo klasi"cno transcendentno ena"cbo
	    \[x=\tan(x).\]
	    \begin{enumerate}
		\item Z bisekcijo preveri, da ima ena"cba re"sitev na intervalu $(3\pi/2,5\pi/2)$ in jo izra"cunaj na 1 decimalko natan"cno.
		\item S tangentno metodo poišči rešitev enačbe na intervalu $(5\pi/2,7\pi/2)$ na 3 decimalke natančno. Pazi pri izbiri začetnega približka. 
		\item Navadno iteracijo lahko uporabimo, če enačbo $x=\tan(x)$ preoblikujemo v enačbo
		    \[x=\tan^{-1}(x),\]
		    pri čemer moramo paziti, katero vejo večlične funkcije $\tan^{-1}(x)$ uporabimo. 
		    Z navadno iteracijo poišči rešitev enačbe na intervalu $(13\pi/2,15\pi/2)$ na 3 decimalke natanko.
		\item[Dodatek] Enačba izgleda kot nalašč za navadno iteracijo(metodo fiksne točke). Prepričaj se, da navadna iteracija dana z rekurzivno formulo
		    \[x_{n+1}=\tan(x_n)\]
		    ne konvergira za noben začetni približek.
	    \end{enumerate}
        \item  
	    Dan je začetni problem za diferencialno enačbo 
	    \[y'(x)+y(x)=x\]
	    in začetni pogoj $y(1)=1$.
	    
	    \begin{enumerate}
		\item Izračunaj $y(2)$ z Eulerjevo metodo s koraki $h=1$ in h=$0.5$. 
		\item Vrednost $y(2)$ izračunaj še z trapezno metodo s korakoma $h=1$ in $h=0.5$.
		\item Določi globalne in lokalne napake  za približke izračunane v 
		    prejšnjih točkah, če veš, da je 
		    $x-1+Ce^{-x}$ 
		    splošna rešitev diferencialne enačbe $y'(x)+y(x)=x$.
	    \end{enumerate}
\end{enumerate}
