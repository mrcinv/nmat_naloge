

\begin{center}
  IZPIT IZ NUMERI"CNE MATEMATIKE\\
  5. september 2003
\end{center}
\vspace{2cm}

\begin{enumerate}
  \item Naj bo $A\in\RR^{n\times n}$ simetri"cna,
  pozitivno definitna matrika. Denimo, da poznate 
  zgornje trikotno matriko
  $R\in\RR^{n\times n}$, za katero je $A=R^TR$ (razcep 
  Choleskega).
  \begin{itemize}
    \item[a)] Zapi"site u"cinkovit algoritem za re"sevanje
    sistema linearnih ena"cb $A^2\mathbf{x}=\mathbf{b}$, kjer
    je $\mathbf{b}\in\RR^{n}$ dani vektor desnih strani.

    \item[b)] Pre"stejte "stevilo potrebnih operacij
    (mno"zenj in deljenj) iz to"cke a).
 
   \item[c)] Uporabite algoritem iz to"cke a) za
    re"sevanje sistema $A^4\mathbf{x}=\mathbf{b}$.
   
   \item[d)] {\sl Dodatek}: Koliko operacij je potrebnih za
   re"sevanje sistema $A^n\mathbf{x}=\mathbf{b}$, kjer je
   $n\in\NN$?
  \end{itemize}
  
  \item Dolo"ceni integral funkcije ra"cunamo 
  z navadno trapezno formulo, torej
  $$\int_{a}^{a+h}f(x)\,dx=\frac{h}{2}
  \left(f(a)+f(a+h)\right)+R(f),$$
  kjer je 
  $$R(f)=-\frac{h^3}{12}f''(\xi),\quad \xi\in[a,a+h].$$
  \begin{itemize}
    \item[a)] Ocenite, kolik"sno najve"cjo absolutno
    napako lahko naredite
    pri ra"cunanju integrala
    $$\int_{0}^{0.1}e^{-x}\,dx,$$
    "ce za korak vzamete $h=0.1$.
    \item[b)] Izra"cunajte to"cno absolutno napako
    iz to"cke a).
    \item[c)] Kolik"sna je napaka, "ce integral iz to"cke
    a) ra"cunate s sestavljenim trapeznim pravilom s korakom
    $h=0.025$?
  \end{itemize}
\end{enumerate}

