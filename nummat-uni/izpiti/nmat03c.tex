\documentclass[12pt,a4paper]{article}
\usepackage[slovene]{babel}
\usepackage{amsfonts,graphicx}
% Mno"zice realnih, naravnih in kompleksnih "stevil
\def\RR{\mathbb{R}}
\def\NN{\mathbb{N}}
\def\CC{\mathbb{C}}

\newtheorem{definicija}{Definicija}
\newtheorem{primer}{Primer}
% Odebeljene "crke v math na"cinu 
% (povezto po L.L. Schumaker - makro
% za St. Malo Proceedings
\def\bfm#1{{\dimen0=.01em\dimen1=.009em\makebold{$#1$}}}

\def\makebold#1{\mathord{\setbox0=\hbox{#1}%
       \copy0\kern-\wd0%
       \raise\dimen1\copy0\kern-\wd0%
       {\advance\dimen1 by \dimen1\raise\dimen1\copy0}\kern-\wd0%
       \kern\dimen0\raise\dimen1\copy0\kern-\wd0%
       {\advance\dimen1 by \dimen1\raise\dimen1\copy0}\kern-\wd0%
       \kern\dimen0\raise\dimen1\copy0\kern-\wd0%
       {\advance\dimen1 by \dimen1\raise\dimen1\copy0}\kern-\wd0%
       \kern\dimen0\raise\dimen1\copy0\kern-\wd0%
       \kern\dimen0\box0}}


\begin{document}

\begin{center}
  IZPIT IZ NUMERI"CNE MATEMATIKE\\
  5. september 2003
\end{center}
\vspace{2cm}

\begin{enumerate}
  \item Naj bo $A\in\RR^{n\times n}$ simetri"cna,
  pozitivno definitna matrika. Denimo, da poznate 
  zgornje trikotno matriko
  $R\in\RR^{n\times n}$, za katero je $A=R^TR$ (razcep 
  Choleskega).
  \begin{itemize}
    \item[a)] Zapi"site u"cinkovit algoritem za re"sevanje
    sistema linearnih ena"cb $A^2\bfm{x}=\bfm{b}$, kjer
    je $\bfm{b}\in\RR^{n}$ dani vektor desnih strani.

    \item[b)] Pre"stejte "stevilo potrebnih operacij
    (mno"zenj in deljenj) iz to"cke a).
 
   \item[c)] Uporabite algoritem iz to"cke a) za
    re"sevanje sistema $A^4\bfm{x}=\bfm{b}$.
   
   \item[d)] {\sl Dodatek}: Koliko operacij je potrebnih za
   re"sevanje sistema $A^n\bfm{x}=\bfm{b}$, kjer je
   $n\in\NN$?
  \end{itemize}
  
  \item Dolo"ceni integral funkcije ra"cunamo 
  z navadno trapezno formulo, torej
  $$\int_{a}^{a+h}f(x)\,dx=\frac{h}{2}
  \left(f(a)+f(a+h)\right)+R(f),$$
  kjer je 
  $$R(f)=-\frac{h^3}{12}f''(\xi),\quad \xi\in[a,a+h].$$
  \begin{itemize}
    \item[a)] Ocenite, kolik"sno najve"cjo absolutno
    napako lahko naredite
    pri ra"cunanju integrala
    $$\int_{0}^{0.1}e^{-x}\,dx,$$
    "ce za korak vzamete $h=0.1$.
    \item[b)] Izra"cunajte to"cno absolutno napako
    iz to"cke a).
    \item[c)] Kolik"sna je napaka, "ce integral iz to"cke
    a) ra"cunate s sestavljenim trapeznim pravilom s korakom
    $h=0.025$?
  \end{itemize}
\end{enumerate}
\end{document}
