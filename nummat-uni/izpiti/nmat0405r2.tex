

\begin{center}
  {\large Re"sitve drugega izpita iz UVNM,\\
    5.5.2005\\
    }
\end{center}

\begin{enumerate}

  \item S tangentno metodo ra"cunate najmanj"so ni"clo
      {\sl Laguerreovega} polinoma $L_3$. Naj bo za"cetni pribli"zek
      $x_0=0$. Zapi"site pribli"zek $x_2$ na $4$ decimalna mesta.
      Laguerreovi polinomi so dani rekurzivno takole
      \begin{align*}
        L_0(x)&=1,\\
        L_1(x)&=1-x,\\
        (n+1)\,L_{n+1}(x)&=(2\,n+1-x)\,L_n(x)-n\,L_{n-1}(x).
      \end{align*}
      Dolo"cite red konvergence metode pri iskanju te ni"cle.\\
      {\sl Namig: Najprej izra"cunajte koeficiente polinoma $L_3$.}
  
   {\bf Re"sitev}: Iz rekurzivne formule dobimo
   $$L_3(x)=\frac{1}{6}(-x^3+9\,x^2-18\,x+6).$$
   Torej lahko iskanje ni"cle $L_3$ poenostavimo v iskanje ni"cle
   funkcije $f(x)=x^3-9\,x^2+18\,x-6$.
   Tangentna metoda se tedaj glasi
   $$x_{n+1}=x_n-\frac{f(x_n)}{f'(x_n)}=x_n-
   \frac{x_n^3-9\,x_n^2+18\,x_n-6}{3\,x_n^2-18\,x_n+18}.$$
   "Ce je $x_0=0$, dobimo $x_2=0.4114$.\\
   Red konvergence je $2$, saj je ni"cla $\alpha$ enostavna
   ($f$ ima tri razli"cne ni"cle) in $f''(\alpha)\neq 0$.
  
  \item {\sl Gauss--Chebysheva} kavadraturna formula
      se glasi
      $$\int_{-1}^{1}\frac{f(x)}{\sqrt{1-x^2}}\,dx=
        \sum_{k=1}^n \alpha_k\,f\left(\cos\left(\frac{2k-1}{2n}\pi\right)\right)+
        \frac{2\,\pi}{2^{2n}(2\,n)!}f^{(2n)}(\xi),$$ 
      kjer je $\xi\in[-1,1]$. 
      Dolo"cite konstanti $\alpha_1$ in $\alpha_2$ ter 
      izra"cunajte numeri"cni pribli"zek
      za integral
      $$\int_{-1}^{1}\frac{e^{x}}{\sqrt{1-x^2}}\,dx$$
      v primeru $n=2$. S pomo"cjo zgoraj zapisane formule
      ocenite absolutno napako pri izra"cunu.

    
    {\bf Re"sitev}: Konstanti $\alpha_1$ in $\alpha_2$ izra"cunamo tako, da
    za $f$ zaporedoma izberemo $1$ in $x$ ter zahtevamo, da je formula to"cna.
    Dobimo sistem
    \begin{align*}
    	\alpha_1+\alpha_2&=\arcsin{x}\vert_{-1}^1=\pi,\\
    	\alpha_1\frac{\sqrt{2}}{2}-\alpha_2\frac{\sqrt{2}}{2}&=0,
    \end{align*}
    katerega re"sitev je $\alpha_1=\alpha_2=\pi/2$.\\
    Numeri"cni pribli"zek za integral je torej
    $$\int_{-1}^{1}\frac{e^{x}}{\sqrt{1-x^2}}\,dx\approx
    \frac{\pi}{2}\left(e^{\frac{\sqrt{2}}{2}}+e^{-\frac{\sqrt{2}}{2}}\right)=
    3.9603.
    $$
    Iz Gauss--Chebysheve formule dobimo oceno za absolutno napako
    \begin{align*}
    	&\left|\int_{-1}^{1}\frac{e^{x}}{\sqrt{1-x^2}}\,dx-
        \sum_{k=1}^2 \alpha_k\,e^{\cos\left(\frac{2k-1}{4}\pi\right)}
       \right|\leq \frac{2\,\pi}{2^{4}(4)!}\max_{\xi\in[-1,1]}
       (e^{x})^{(4)}\\
       &<\frac{2\,\pi}{2^{4}(4)!}e^1=0.0445.
    \end{align*}
    
  \item Tabelo podatkov $(x_i,f_i)$, $i=1,2,\dots,n$, 
      aproksimirate s konstanto po metodi
      najmanj"sih kvadratov. Preverite, da je konstanta enaka povpre"cni vrednosti
      "stevil $f_i$.  
    
    {\bf Re"sitev}: Vzemimo za aproksimacijsko funkcijo $f(x)=c$, kjer
     je $c$ iskana konstanta.
     Minimizirati moramo funkcijo
     $$E(c)=\sum_{i=1}^n(f(x_i)-f_i)^2=
     \sum_{i=1}^n(c-f_i)^2.
     $$
     Ko njen odvod postavimo na $0$, dobimo pogoj
     $$c=\frac{1}{n}\sum_{i=1}^n f_i,$$ 
     kar je ravno povpre"cna vrednost.
    	
  
  \item 
    Nihanje matemati"cnega nihala opi"semo z eksaktnim  
    $$\ddot\varphi(t)+\frac{g}{\ell}\sin(\varphi(t))=0, \quad \varphi(0)=\varphi_0,\ 
      \dot\varphi(0)=\dot\varphi_0$$
    in poenostavljenim 
     $$\ddot\varphi(t)+\frac{g}{\ell}\varphi(t)=0, \quad \varphi(0)=\varphi_0,\ 
      \dot\varphi(0)=\dot\varphi_0$$
    za"cetnim problemom. 
    Izra"cunajte razliko kotnih hitrosti, ki ju dobimo z re"sevanjem prvega in drugega
    problema z Eulerjevo metodo po $0.5s$. Korak naj bo $0.25s$, $\ell=0.1m$,
    $\varphi_0=0.5$ in $\dot\varphi_0=0s^{-1}$.
    
    {\bf Re"sitev}: Obe diferencialni ena"cbi prevedemo na sistem dveh 
    prvega reda
    \begin{align*}
    	\bfm{Y}^{'}&:=\bfm{F}_1(t,\bfm{Y}):=[\bfm{Y}(2),-\frac{g}{\ell}\,\sin(\bfm{Y}(1))]^T,\\
	\bfm{Y}^{'}&:=\bfm{F}_2(t,\bfm{Y}):=[\bfm{Y}(2),-\frac{g}{\ell}\,\bfm{Y}(1)]^T, 	
    \end{align*}
    kjer je $\bfm{Y}=[\varphi,\dot\varphi]^T$.
    Eulerjeva metoda se potem glasi
    $$\bfm{Y}_{i+1}=\bfm{Y}_i+h\bfm{F}_i(t_i,\bfm{Y}_i).$$
    Z danimi podatki iz naloge pridemo do pribli"zkov za kotno hitrost.
    V prvem primeru $\dot\varphi(0.5)\approx -0.2349$ in 
    v drugem$\dot\varphi(0.5)\approx -0.2450$.
    Torej je absolutna vrednost razlike enaka $0.0101$.
    
    
\end{enumerate}
