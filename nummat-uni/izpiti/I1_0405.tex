\documentclass[12pt,a4paper]{article}

\usepackage[slovene]{babel}
\usepackage{amsfonts}
%Real and complex numbers
\def\RR{\mathbb{R}}
\def\CC{\mathbb{C}}
\def\NN{\mathbb{N}}
% Boldface math
\def\bfm#1{{\dimen0=.01em\dimen1=.009em\makebold{$#1$}}}

\def\makebold#1{\mathord{\setbox0=\hbox{#1}
       \copy0\kern-\wd0
       \raise\dimen1\copy0\kern-\wd0
       {\advance\dimen1 by \dimen1\raise\dimen1\copy0}\kern-\wd0
       \kern\dimen0\raise\dimen1\copy0\kern-\wd0
       {\advance\dimen1 by \dimen1\raise\dimen1\copy0}\kern-\wd0
       \kern\dimen0\raise\dimen1\copy0\kern-\wd0
       {\advance\dimen1 by \dimen1\raise\dimen1\copy0}\kern-\wd0
       \kern\dimen0\raise\dimen1\copy0\kern-\wd0
       \kern\dimen0\box0}}


\pagestyle{empty}

\begin{document}

\begin{center}
  {\large UVOD V NUMERI"CNE METODE\\
    1. izpit, 7.2.2005\\
    }
\end{center}
\vspace{2cm}

\begin{enumerate}

  \item Dolo"cite najmanj"so stopnjo polinoma s katerim
    lahko interpoliramo naslednjo tabelo podatkov
    $$\begin{array}{r|rrrrrr}
      x & -2 & -1 & 0 & 1 & 2 & 3\\ \hline
      y & -5 & 1  & 1 & 1 & 7 & 25
      \end{array}.
    $$
    Prepri"cajte se, da ima polinom najni"zje stopnje, ki interpolira
    zgornjo tabelo, eno samo ni"clo na $[-2,3]$. To ni"clo izra"cunajte s
    tangentno metodo na dve decimalni mesti natan"cno.
  
  \item V naslednji tabeli so podatki hitrosti avtomobila ($v$[km/h])
    v prvih dveh sekundah vo"znje
    $$\begin{array}{l|rrrrr}
    v & 40 & 46 & 51 & 55 & 59\\ \hline
    t & 0  & 0.5 & 1 & 1.5 & 2
    \end{array}.$$
    S trapeznim in Simpsonovim pravilom izra"cunajte dol"zino poti 
    (v me\-trih), ki jo je prevozil avtomobil v prvih dveh sekundah. Nato 
    z vrednostmi hitrosti ob "casih $0.5s$, $1s$ in $1.5s$ "cim bolje ocenite
    pospe"sek avtomobila po eni sekundi.

  \item Naj bo 
    $$A=\left(
      \begin{array}{ccccccc}
        x & 0 & 0 & \cdots & 0 & 0 & a_0\\
        -1& x & 0 & \cdots & 0 & 0 & a_1\\
        0 & -1& x & \cdots & 0 & 0 & a_2\\
        \vdots & \vdots & \ddots & \ddots & \vdots & \vdots & \vdots\\
        0 & 0 & \cdots & \cdots & -1 & x & a_{n-1}\\
        0 & 0 & \cdots & \cdots & 0 & -1 & x
      \end{array}\right) .
    $$
      Izra"cunajte determinanto matrike $A$ tako, da najprej
      izra"cunate njen $LU$ razcep in upo"stevate dejstvo
      $\det A=\det U=\prod_{i=1}^{n+1}u_{ii}$.
  \item Nekdo je re"seval diferencialno ena"cbo $y''(x)=-x\,y(x)$ 
    na intervalu $[0,0.2]$ z Eulerjevo metodo s korakom $h=0.1$.
    Za vrednost funkcije $y(0.2)$ je dobil numeri"cni pribli"zek $0.2$,
    za odvod $y'(0.2)$ pa $0.999$. Iz\-ra\-"cu\-najte vrednosti 
    $y(0)$ in $y'(0)$?
\end{enumerate}
\end{document} 
