\documentclass[12pt,a4paper]{article}

\usepackage[slovene]{babel}
\usepackage{amsfonts}
%Real and complex numbers
\def\RR{\mathbb{R}}
\def\CC{\mathbb{C}}
\def\NN{\mathbb{N}}
% Boldface math
\def\bfm#1{{\dimen0=.01em\dimen1=.009em\makebold{$#1$}}}

\def\makebold#1{\mathord{\setbox0=\hbox{#1}
       \copy0\kern-\wd0
       \raise\dimen1\copy0\kern-\wd0
       {\advance\dimen1 by \dimen1\raise\dimen1\copy0}\kern-\wd0
       \kern\dimen0\raise\dimen1\copy0\kern-\wd0
       {\advance\dimen1 by \dimen1\raise\dimen1\copy0}\kern-\wd0
       \kern\dimen0\raise\dimen1\copy0\kern-\wd0
       {\advance\dimen1 by \dimen1\raise\dimen1\copy0}\kern-\wd0
       \kern\dimen0\raise\dimen1\copy0\kern-\wd0
       \kern\dimen0\box0}}


\pagestyle{empty}

\begin{document}

\begin{center}
  {\large UVOD V NUMERI"CNE METODE\\
    2. kolokvij, 31.1.2005\\
    }
\end{center}

\begin{enumerate}

  \item Naj bo $A\in\RR^{n\times n}$ tridiagonalna matrika. 
    Napi"site algoritem, ki v "casu ${\cal O}(n)$ izra"cuna
    $LU$ razcep matrike $A$. Utemeljite, zakaj je va"s
    algoritem res ustrezne "casovne zahtevnosti.\\
    Predlagani algoritem uporabite za izra"cun $LU$ razcepa matrike
    $$A=\left(
      \begin{array}{ccc}
        1 & 2 & 0\\
        1 & 3 & 2\\
        0 & 1 & 3
      \end{array}
      \right).$$
        
  
  \item Tabelo podatkov
   $$\begin{array}{c|rrrrr}
     x_i & -2 & -1 & 0 & 1 & 2\\ \hline
     y_i &  2 &  1 & 0 & 1 & 2
     \end{array}
   $$
   zaradi o"citne simetrije aproksimiramo s parabolo oblike
    $p(x)=\alpha\,x^2+\beta$. 
    Izra"cunajte tisti par parametrov $\alpha$ in
    $\beta$, pri katerem je dose"zena vrednost
    $$\min_{\alpha,\beta\in\RR}\sum_{i=1}^5(p(x_i)-y_i)^2.$$
  
  \item Naj bo 
    $$ 
    A=\pmatrix{1   & 0.1 & 0.2\cr
               0.1 &  2  & -0.1\cr
               x   &  0.2& 3 }.
    $$
    Naslednje probleme re"site {\bf brez ra"cunanja} lastnih vrednosti:
    \begin{itemize}
      \item[a)] Preverite, da matrika $A$ za $x\in(-2.8,2.8)$ ne more imeti 
        negativnih lastnih vrednosti. 
      \item[b)] Za kak"sne $x\in\RR$ gotovo nima nobene dvojne
        lastne vrednosti? 
      \item[c)] "Cim bolje ocenite najmanj"so lastno vrednost v primeru, 
        ko je $x=0$.
    \end{itemize}
    {\sl Namig: Uporabite Gershgorinov izrek.}
  
  \item Re"sujete {\sl Besselovo} diferencialno ena"cbo
    $$x^2\,y''(x)+x\,y'(x)+x^2\,y(x)=0, \quad y(0)=1,\ y'(0)=0.$$
    Z Eulerjevo metodo in korakom $h=0.1$ ocenite vrednost
    $y'(0.2)$.\\
    {\sl Namig: $y''(0)=-\frac{1}{2}$}.

\end{enumerate}
\end{document} 
