\documentclass[12pt,a4paper]{article}

\usepackage[slovene]{babel}

\def\bfm#1{{\dimen0=.01em\dimen1=.009em\makebold{$#1$}}}
 
\def\makebold#1{\mathord{\setbox0=\hbox{#1}%
       \copy0\kern-\wd0%
       \raise\dimen1\copy0\kern-\wd0%
       {\advance\dimen1 by \dimen1\raise\dimen1\copy0}\kern-\wd0%
       \kern\dimen0\raise\dimen1\copy0\kern-\wd0%
       {\advance\dimen1 by \dimen1\raise\dimen1\copy0}\kern-\wd0%
       \kern\dimen0\raise\dimen1\copy0\kern-\wd0%
       {\advance\dimen1 by \dimen1\raise\dimen1\copy0}\kern-\wd0%
       \kern\dimen0\raise\dimen1\copy0\kern-\wd0%
       \kern\dimen0\box0}}   


\pagestyle{empty}

\begin{document}
\begin{center}
  IZPIT IZ NUMERI"CNE MATEMATIKE\\
  14. junij 2002
\end{center}

\begin{enumerate}

  \item Vemo, da za "stevila $x$, ki so po absolutni vrednosti dovolj majhna,
    pribli"zno velja
    $$\sin{x}\approx x-\frac{x^3}{3!}+\frac{x^5}{5!}.$$
    \begin{itemize}
       \item[a)] Omenjeno zvezo lahko uporabimo tudi za ra"cunanje vrednosti
        $\arcsin{y}$, pri dovolj majhnem $y$. V ta namen re"sujemo
        ena"cbo
        $$x-\frac{x^3}{3!}+\frac{x^5}{5!}=y.$$
        Napi"site algoritem, ki ra"cuna zaporedne pribli"zke za $\arcsin{y}$
        tako, da ena"cbo re"sujete z Newtonovo iteracijo.
       \item[b)]
        Izra"cunajte drugi pribli"zek $x_2$ za $\arcsin{0.2}$ z metodo iz
        to"cke a). Za"cetni pribli"zek naj bo $x_0=0.1$.
       \item[c)] Kolik"sna je absolutna napaka rezultata iz to"cke b)?
    \end{itemize}

  \item Hermitov interpolacijski polinom tretje stopnje ima to lastnost, da 
    interpolira vrednost funkcije in njenega odvoda v dveh izbranih to"ckah.
    Dana je naslednja tabela podatkov
    $$
        \begin{array}{|l||l|l|}
          \hline
          x & 0 & 1 \\ \hline\hline
          f(x) & 0 & \alpha \\ \hline
          f'(x) & 0 & 1\\ \hline
        \end{array}
    $$
     
     \begin{itemize}
      \item[a)] Poi"s"cite Hermitov polinom v primeru, ko je $\alpha=0$.
      \item[b)] Kak"sen mora biti $\alpha$, da bo stopnja interpolacijskega 
      polinoma kve"cjemu $2$?
      \item[c)] Ali interpolacijski polinom obstaja za vsak $\alpha$?
     \end{itemize}   
\end{enumerate}


\end{document}
         
