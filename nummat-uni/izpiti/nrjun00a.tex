\documentclass[12pt,a4paper]{article}

\usepackage[slovene]{babel}

\pagestyle{empty}

\begin{document}

  \begin{center}
    {\Large IZPIT IZ NUMERI"CNE MATEMATIKE}\\
    15.6.2000
  \end{center}
\begin{enumerate}

  \item Funkcija $f$ zado"s"ca pogojem: $f(0)=f_0$, $f(x_1)=f_1$
  in $f(1)=f_2$, $0< x_1<1$. Na osnovi teh podatkov 
  i"s"cemo interpolacijski
  polinom $p(x)=a\,x^2+b\,x+c$, ki interpolira
  $f$ v to"ckah $0$, $x_1$ in $1$. Polinom $p$ dolo"cimo
  tako, da re"simo sistem linearnih ena"cb
  $$p(0)=f_0,\quad p(x_1)=f_1,\quad p(1)=f_1.$$

  \begin{enumerate}
   
     \item Poi"s"ci interpolacijski polinom na omenjeni na"cin
     v primeru, ko je $x_1=1/3$, $f_0=0$, $f_1=2/9$ in
     $f_2=0$.

     \item Kak"sna je ob"cutljivost matrike sistema linearnih
     ena"cb iz to"cke (a) v maksimum normi?\\
     Spomni se, da je ob"cutljivost matrike $A$ v maksimum normi
     definirana kot ${\rm cond}_\infty(A)=\|A\|_\infty\,
     \|A^{-1}\|_\infty$.

     \item  Kak"sen mora biti $x_1$, da bo ob"cutljivost matrike
     dobljenega linearnega sistema v maksimum normi najmanj"sa?
     Izra"cunaj najmanj"so ob"cutljivost!
  \end{enumerate}

  \item Re"sujemo ena"cbo $f(x)=0.1$, kjer je 
        $$f(x)=\int_{0}^{x}e^{-t^2}\,dt.$$
        
        \begin{enumerate}
         
           
          \item Zapi"si algoritem, ki z Newtonovo metodo ra"cuna re"sitev
                zgornje ena"cbe.
          \item S pomo"cjo algoritma iz to"cke (a) izra"cunaj prva dva 
                pribli"zka, "ce je $x_0=0.2$, integral pa ra"cuna"s s
                trapezno metodo s korakom $h=x_n$.
          
          \item Prva dva pribli"zka izra"cunaj "se tako, da integral 
                ra"cuna"s z enostavno Simpsonovo formulo.
        \end{enumerate}  

\end{enumerate}


\end{document}
         