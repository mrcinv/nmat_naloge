\documentclass[12pt,a4paper]{article}
\usepackage[slovene]{babel}
\usepackage{amsfonts,graphicx}
% Mno"zice realnih, naravnih in kompleksnih "stevil
\def\RR{\mathbb{R}}
\def\NN{\mathbb{N}}
\def\CC{\mathbb{C}}
\addtolength{\textheight}{4cm}
\addtolength{\voffset}{-2cm}

\newtheorem{definicija}{Definicija}
\newtheorem{primer}{Primer}
% Odebeljene "crke v math na"cinu 
% (povezto po L.L. Schumaker - makro
% za St. Malo Proceedings
\def\bfm#1{{\dimen0=.01em\dimen1=.009em\makebold{$#1$}}}

\def\makebold#1{\mathord{\setbox0=\hbox{#1}%
       \copy0\kern-\wd0%
       \raise\dimen1\copy0\kern-\wd0%
       {\advance\dimen1 by \dimen1\raise\dimen1\copy0}\kern-\wd0%
       \kern\dimen0\raise\dimen1\copy0\kern-\wd0%
       {\advance\dimen1 by \dimen1\raise\dimen1\copy0}\kern-\wd0%
       \kern\dimen0\raise\dimen1\copy0\kern-\wd0%
       {\advance\dimen1 by \dimen1\raise\dimen1\copy0}\kern-\wd0%
       \kern\dimen0\raise\dimen1\copy0\kern-\wd0%
       \kern\dimen0\box0}}
\pagestyle{empty}

\begin{document}

\begin{center}
  IZPIT IZ NUMERI"CNE MATEMATIKE\\
  30. avgust 2004
\end{center}
\vspace{1cm}

\begin{enumerate}
  \item Dana je tabela podatkov
    $$\begin{array}{|r|r|r|r|r|r|}
           \hline
	    	x & -2 & -1 & 0 & 1 & 2 \\ \hline
		y & -5 &  0 & 1 & y_3 & y_4\\ \hline
	\end{array}
	$$
	\begin{itemize}
		\item[a)] Kak"sna mora biti zveza med $y_3$ in $y_4$,
		da bo tabelo mogo"ce interpolirati s kubi"cnim polinomom?
		\item [b)]Dolo"cite tak par "stevil $y_3$ in $y_4$,
		da je tabelo mogo"ce interpolirati s kubi"cnim
		polinomom z vodilnim koeficientom $1$.
		\item[c)] Kolik"sna je vrednost interpolacijskega polinoma
		iz b) pri $x=\pi/2$?
								
  \end{itemize}
  \item I"s"cemo re"sitev za"cetnega problema $y'=y-x$, $y(0)=0$.
    \begin{itemize}
    	\item[a)] Poi"s"cite to"cno re"sitev za"cetnega problema.
	\item[b)] Na intervalu $[0,1]$ poi"s"cite numeri"cno re"sitev 
	z Adams-Bashforth-Moultonovo prediktor-korektor metodo "cetrtega 
	reda. Dodatne vrednosti izra"cunajte z Runge-Kutta metodo "cetrtega 
	reda. Za korak vzemite $0.25$.
	\item[c)] Kolik"sna je globalna napaka pri $x=1$?
    \end{itemize}
\end{enumerate}


\end{document}
