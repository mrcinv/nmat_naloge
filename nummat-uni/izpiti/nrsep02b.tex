\documentclass[12pt,a4paper]{article}

\usepackage[slovene]{babel}

\def\bfm#1{{\dimen0=.01em\dimen1=.009em\makebold{$#1$}}}
 
\def\makebold#1{\mathord{\setbox0=\hbox{#1}%
       \copy0\kern-\wd0%
       \raise\dimen1\copy0\kern-\wd0%
       {\advance\dimen1 by \dimen1\raise\dimen1\copy0}\kern-\wd0%
       \kern\dimen0\raise\dimen1\copy0\kern-\wd0%
       {\advance\dimen1 by \dimen1\raise\dimen1\copy0}\kern-\wd0%
       \kern\dimen0\raise\dimen1\copy0\kern-\wd0%
       {\advance\dimen1 by \dimen1\raise\dimen1\copy0}\kern-\wd0%
       \kern\dimen0\raise\dimen1\copy0\kern-\wd0%
       \kern\dimen0\box0}}   


\pagestyle{empty}

\begin{document}
\begin{center}
  IZPIT IZ NUMERI"CNE MATEMATIKE\\
  20. september 2002
\end{center}

\begin{enumerate}

  \item Naravni logaritem "stevila $a=1+\epsilon$, kjer je $|\epsilon|$
    majhno "stevilo, lahko izra"cunamo pribli"zno tako, da
    re"simo ena"cbo
    $$\frac{1+\frac{x}{2}+\frac{x^2}{8}+\frac{x^3}{48}}
      {1-\frac{x}{2}+\frac{x^2}{8}-\frac{x^3}{48}}=a.
    $$
    \begin{itemize}
      \item[a)] Zgornjo ena"cbo preoblikujte v polinomsko.
      \item[b)] Za ena"cbo iz to"cke a) napi"site algoritem, ki
        ra"cuna njene re"sitve po Newtonovi metodi.
      \item[c)] Z metodo iz to"cke b) izra"cunajte pribli"zek
	za $\ln(1.1)$. Za"cetni pribli"zek naj bo $x_0=0$.
        Dovolj je, "ce izra"cunate "se pribli"zka $x_1$ in $x_2$.
      \item[d)] Kolik"sna je absolutna napaka rezultata iz c)?
    \end{itemize}
  \item  
\end{enumerate}


\end{document}
         
