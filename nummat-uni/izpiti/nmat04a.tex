\documentclass[12pt,a4paper]{article}
\usepackage[slovene]{babel}
\usepackage{amsfonts,graphicx}
% Mno"zice realnih, naravnih in kompleksnih "stevil
\def\RR{\mathbb{R}}
\def\NN{\mathbb{N}}
\def\CC{\mathbb{C}}
\addtolength{\textheight}{4cm}
\addtolength{\voffset}{-2cm}

\newtheorem{definicija}{Definicija}
\newtheorem{primer}{Primer}
% Odebeljene "crke v math na"cinu 
% (povezto po L.L. Schumaker - makro
% za St. Malo Proceedings
\def\bfm#1{{\dimen0=.01em\dimen1=.009em\makebold{$#1$}}}

\def\makebold#1{\mathord{\setbox0=\hbox{#1}%
       \copy0\kern-\wd0%
       \raise\dimen1\copy0\kern-\wd0%
       {\advance\dimen1 by \dimen1\raise\dimen1\copy0}\kern-\wd0%
       \kern\dimen0\raise\dimen1\copy0\kern-\wd0%
       {\advance\dimen1 by \dimen1\raise\dimen1\copy0}\kern-\wd0%
       \kern\dimen0\raise\dimen1\copy0\kern-\wd0%
       {\advance\dimen1 by \dimen1\raise\dimen1\copy0}\kern-\wd0%
       \kern\dimen0\raise\dimen1\copy0\kern-\wd0%
       \kern\dimen0\box0}}
\pagestyle{empty}

\begin{document}

\begin{center}
  IZPIT IZ NUMERI"CNE MATEMATIKE\\
  7. junij 2004
\end{center}
\vspace{1cm}

\begin{enumerate}
  \item Naj bo $f(x)=\cos{x}-\alpha\,x$, $\alpha\in\RR$.
  \begin{itemize}
    \item[a)] Dolo"cite vsa realna "stevila $\alpha$, za katera
      ima ena"cba $f(x)=0$ vsaj eno re"sitev na $[0,1]$.
    \item[b)] Re"sitev ena"cbe $f(x)=0$ poi"s"cite z Newtonovo metodo
      v primeru $\alpha=1$.
      Za"cetni pribli"zek izberite sami, ni"cla naj bo izra"cunana
      vsaj na tri decimalna mesta natan"cno.
    \item[c)] Naj bo $|\alpha|<1$. Poi"s"cite vsaj en za"cetni pribli"zek
      $x_0$ pri katerem Newtonova metoda za re"sevanje ena"cbe
      $f(x)=0$ odpove na {\bf prvem} koraku.  
  \end{itemize}
  \item Vrv pritrdimo v to"ckah $\bfm{T}_1(0,1)$ in $\bfm{T}_2(1,2)$ tako,
    da prosto visi. Koordinate nekaj to"ck na vrvi lahko 
    preberemo v naslednji tabeli:
    \begin{center}
    \begin{tabular}{r|cccc}
        $x$ & 0 & 0.3333 & 0.6667 & 2\\ \hline
        $y$ & 1.0000 & 1.1854 & 1.3801 & 2.0000
      \end{tabular}
    \end{center}
    
    Obliko vrvi modeliramo s funkcijo $f(x)=a\,{\rm ch}(x)+b$ po metodi
    najmanj"sih kvadratov.
    \begin{itemize}
     \item[a)] Zapi"site algoritem za izra"cun parametrov $a$ in $b$.
     \item[b)] Izra"cunajte parametra $a$ in $b$ za dano tabelo z 
       algoritmom iz to"cke a.
     \item[c)] Obliko vrvi iz zgornje tabele sedaj modeliramo s funkcijo 
       $$g(x)=c\,(e^{x}+e^{-x})+d$$ 
       po metodi najmanj"sih kvadratov.
       Kak"sna sta sedaj parametra $c$ in $d$?\\
       {\sl Namig: poskusite uporabiti rezultat iz to"cke b.}
    \end{itemize}
\end{enumerate}


\end{document}
