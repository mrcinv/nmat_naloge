
\begin{center}
  IZPIT IZ NUMERI"CNE MATEMATIKE\\
  21. september 2001
\end{center}

\vspace{1.5cm}
\begin{enumerate}
 \item V kraji"s"cih intervala $[a,a+h]$ poznamo vrednosti 
    funkcije $f$ in njenih odvodov. Izpeljati "zelimo 
    integracijsko formulo oblike
    $$\int_{a}^{a+h} f(x)\,dx\approx \alpha\,f(a)+
    \beta\,f'(a)+\gamma\,f(a+h)+\delta\,f'(a+h).$$
  \begin{itemize}
    \item[a)] Dolo"cite koeficiente $\alpha$, $\beta$, $\gamma$
      in $\delta$ tako, da bo formula "cim vi"sjega reda (namig:
      predpostavite lahko, da je $a=0$).
    \item[b)] Na podlagi te formule izpeljite "se 
      sestavljeno integracijsko pravilo.
    \item[c)] S pravilom iz to"cke b) izra"cunajte integral
      $$\int_{0}^{0.3} \sin{x}\,dx,$$
      pri koraku $h=0.1$. Kolik"sna je absolutna napaka rezultata?
  \end{itemize}

 \item Re"sujete za"cetni problem
    $$y''(x)+3\,y'(x)-4\,y(x)=0,\quad y(0)=0,\ y'(0)=5.$$
  \begin{itemize}
     \item[a)] Poi"s"cite to"cno re"sitev problema.
     \item[b)] Prevedite diferencialno ena"cbo na sistem dveh
       diferencialnih ena"cb prvega reda.
     \item[c)] Sistem iz to"cke b) re"site z Eulerjevo metodo
       na intevalu $[0,1]$ s korakom $h=0.25$. Kolik"sna je
       globalna napaka v to"cki $x=1$?
     \end{itemize}
\end{enumerate}
