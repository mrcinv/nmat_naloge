
\large
\begin{center}
  {\Large IZPIT IZ NUMERI"CNE MATEMATIKE\\
        15.9.2004
    }
\end{center}
\vspace{1.5cm}
\begin{enumerate}
        \item Za elemente realne matrike $A$ dimenzije $n\times n$ 
          velja: $a_{ij}=0$ za $j>i+1$.
          \begin{itemize}
            \item[a)] Zapi"site en primer take matrike za $n=4$.
            \item[b)] Napi"site {\bf u"cinkovit} algoritem za 
              izra"cun novega pribli"zka Gauss-Seidelove iteracije pri re"sevanju 
	      sistema $A x = b$.
            \item[c)] Pre"stejte "stevilo operacij (mno"zenj in deljenj)
              v to"cki b).
          \end{itemize} 
        \item I"s"cete realne ni"cle polinoma $p(x)=x^3+2x^2+3x+4$.
        \begin{itemize}
                \item[a)] Preverite, da ima $p$ realno ni"clo na
                  intervalu $[-2,-1]$.
                \item[b)] Ni"clo na intervalu $[-2,-1]$ poi"s"cite na
                  eno decimalno mesto na\-tan\-"cno z metodo regula falsi.
                \item[c)] Koliko se relativno spremeni realna ni"cla
                  iz to"cke b), "ce vodilni koeficient polinoma $p$
                  pove"camo za $0.1$?
        \end{itemize} 
\end{enumerate}
\newpage
\begin{center}
  {\Large IZPIT IZ NUMERI"CNE MATEMATIKE\\
        15.9.2004
    }
\end{center}
\vspace{1.5cm}
\begin{enumerate}
        \item Za elemente realne matrike $A$ dimenzije $n\times n$ 
          velja: $a_{ij}=0$ za $j>i+1$.
          \begin{itemize}
            \item[a)] Zapi"site en primer take matrike za $n=4$.
            \item[b)] Napi"site {\bf u"cinkovit} algoritem za 
              izra"cun novega pribli"zka Gauss-Seidelove iteracije pri re"sevanju 
	      sistema $A x = b$.
            \item[c)] Pre"stejte "stevilo operacij (mno"zenj in deljenj)
              v to"cki b).
          \end{itemize} 
        \item I"s"cete realne ni"cle polinoma $p(x)=x^3+2x^2+3x+4$.
        \begin{itemize}
                \item[a)] Preverite, da ima $p$ realno ni"clo na
                  intervalu $[-2,-1]$.
                \item[b)] Ni"clo na intervalu $[-2,-1]$ poi"s"cite na
                  eno decimalno mesto na\-tan\-"cno z metodo regula falsi.
                \item[c)] Koliko se relativno spremeni realna ni"cla
                  iz to"cke b), "ce vodilni koeficient polinoma $p$
                  pove"camo za $0.1$?
        \end{itemize} 
\end{enumerate}
