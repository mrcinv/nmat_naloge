%% LyX 2.0.3 created this file.  For more info, see http://www.lyx.org/.
%% Do not edit unless you really know what you are doing.
\documentclass[slovene]{article}
\usepackage[T1]{fontenc}
\usepackage[latin2]{inputenc}
\usepackage{listings}
\lstset{basicstyle={\footnotesize},
frame=lines,
language=Octave}
\usepackage[a4paper]{geometry}
\geometry{verbose,tmargin=1cm,bmargin=2cm,lmargin=3cm,rmargin=3cm}
\usepackage{amsmath}
\usepackage{esint}
\usepackage{babel}
\begin{document}

\title{Izpit iz Numeri�nih metod 2 za Finan�ne matematike}


\date{28. 6. 2013}
\maketitle
\begin{enumerate}
\item V prostoru $2\times2$ matrik lahko definiramo skalrni produkt s formulo
\[
\langle A,B\rangle=\mathrm{sled}(A^{T}B),
\]
kjer je $\mathrm{sled}\left(\begin{bmatrix}a & b\\
c & d
\end{bmatrix}\right)=a+d$. Poi��i element najbolj�e aproksimacije za matriko
\[
\begin{bmatrix}1 & 2\\
3 & 2
\end{bmatrix}
\]
na podprostor $\mathcal{S}$ simetri�nih matrik s sledjo enako $0$
\[
\mathcal{S}=\{A;\quad A^{T}=A\text{ in }\mbox{\ensuremath{\mathrm{sled}}}(A)=0\}.
\]

\item I��emo ni�lo funkcije
\[
f(x)=x-2^{-x}
\]


\begin{enumerate}
\item Poka�i, da je ni�la na intervalu $[0,1]$ in jo poi��i na 2 decimalke
natan�no.
\item Ni�lo poi��i z inverzno interpolacijo, tako da inverzno funkcijo $f^{-1}(x)$
interpolira� v to�kah $f(0),$ $f\left(\frac{1}{2}\right)$ in $f(1)$
s kvadratnim polinomom. Tako dobljeno ni�lo primerjaj z re�itvijo
iz to�ke $(a)$.
\end{enumerate}
\item Integralski sinus \global\long\def\si{\mathrm{si}}
$\si(x)$ je dan s formulo
\[
\si(x)=\int_{0}^{x}\frac{\sin x}{x}\mathrm{d}x.
\]
Nekatere vrednosti $\si(x)$ so zbrane v tabeli
\[
\begin{array}{c|cccc}
x & 1 & 1.1 & 1.2 & 1.3\\
\hline \si(x) & 0.94608 & 1.02869 & 1.10805 & 1.18396
\end{array}
\]


\begin{enumerate}
\item Vrednosti v tabeli interploliraj z linearnim zlepkom in poi��i $x_{0}$,
za katerega je $\si(x_{0})=1$.
\item S simpsonovo formulo izra�unaj, za koliko se vrednosti $\si(x_{0})$
razlikuje od $1$ za pribli�ek iz prej�nje to�ke.
\item Pribli�ek iz to�ke (a) izbolj�aj z enim korakom Newtonove metode in
zopet preveri, za koliko se vrednost $\si$ razlikuje od $1$.
\end{enumerate}
\item Naj bosta 
\[
Q=\frac{1}{\sqrt{2}}\begin{bmatrix}1 & 0 & 1 & 0\\
0 & 1 & 0 & 1\\
1 & 0 & -1 & 0\\
0 & 1 & 0 & -1
\end{bmatrix}\quad R=\begin{bmatrix}2 & 0 & 1 & 2\\
0 & -3 & 0 & 1\\
0 & 0 & 2 & 1\\
0 & 0 & -1 & 2
\end{bmatrix}
\]
faktorja v realni Schurovi formi matrike $A=QR$. 

\begin{enumerate}
\item Poi��i vse lastne vrednosti matrike in realne lastne vektorje za vsako
od realnih lastnih vrednosti matrike $A$.
\item Ali poten�na metoda konvergira?\end{enumerate}
\end{enumerate}

\end{document}
