%% LyX 2.0.3 created this file.  For more info, see http://www.lyx.org/.
%% Do not edit unless you really know what you are doing.
\documentclass[slovene]{article}
\usepackage[T1]{fontenc}
\usepackage[latin2]{inputenc}
\usepackage[a4paper]{geometry}
\geometry{verbose,tmargin=1cm,bmargin=2cm,lmargin=3cm,rmargin=3cm}
\usepackage{amsmath}
\usepackage{amssymb}
\usepackage{esint}

\makeatletter
%%%%%%%%%%%%%%%%%%%%%%%%%%%%%% User specified LaTeX commands.
\usepackage{babel}


\makeatother

\usepackage{babel}
\begin{document}

\title{Izpit iz Numeri�nih metod 2 za Finan�ne matematike}


\date{28. 6. 2013 }
\maketitle
\begin{enumerate}
\item V prostoru $2\times2$ matrik lahko definiramo skalrni produkt s formulo
\[
\langle A,B\rangle=\mathrm{sled}(A^{T}B),
\]
kjer je $\mathrm{sled}\left(\begin{bmatrix}a & b\\
c & d
\end{bmatrix}\right)=a+d$. Poi��i element najbolj�e aproksimacije po metodi najmanj�ih kvadratov
za matriko 
\[
\begin{bmatrix}1 & 2\\
3 & 2
\end{bmatrix}
\]
na podprostor $\mathcal{S}$ simetri�nih matrik s sledjo enako $0$
\[
\mathcal{S}=\{A;\quad A^{T}=A\text{ in }\mbox{\ensuremath{\mathrm{sled}}}(A)=0\}.
\]

\item Integralski sinus \global\long\def\si{\mathrm{si}}
 $\si(x)$ je dan s formulo 
\[
\si(x)=\int_{0}^{x}\frac{\sin x}{x}\mathrm{d}x.
\]
Nekatere vrednosti $\si(x)$ so zbrane v tabeli 
\[
\begin{array}{c|cccc}
x & 1 & 1.1 & 1.2 & 1.3\\
\hline \si(x) & 0.94608 & 1.02869 & 1.10805 & 1.18396
\end{array}
\]


\begin{enumerate}
\item Vrednosti v tabeli interploliraj z linearnim zlepkom in poi��i $x_{0}$,
za katerega je $\si(x_{0})=1$. 
\item Z osnovno simpsonovo formulo izra�unaj $\si(x_{0})$ za pribli�ek
iz prej�nje to�ke. Na podlagi pribli�ka za $\si(x_{0})$ in ocene
za napako simpsonove formule oceni, za koliko se lahko $\si(x_{0})$
najve� razlikuje od $1$. \emph{(Upo�teva� lahko, da je $||\frac{d^{4}}{dx^{4}}\frac{\sin x}{x}||_{\infty,[1,1.3]}=-20\cos(1)+13\sin(1)$.)} 
\item \emph{(Dodatek 3 to�ke)} Pribli�ek iz to�ke (a) izbolj�aj z enim korakom
Newtonove metode in zopet preveri, za koliko se vrednost $\si$ razlikuje
od $1$. 
\end{enumerate}
\item I��emo re�itev za�etnega problema za NDE 
\begin{equation}
x^{2}y''(x)-2xy'(x)+2y(x)=0\label{eq:cauchy}
\end{equation}
z za�etnimi pogoji $y(1)=0$ in $y'(1)=1$.

\begin{enumerate}
\item Ena�bo \eqref{eq:cauchy} zapi�i v obliki sistema dveh ena�b 1. reda.
\item Poi��i $y(1.1)$ z metodo Runge-Kutta s tabelo
\[
\begin{array}{c|cc}
0 & 0\\
1 & 1 & 0\\
\hline  & \frac{1}{2} & \frac{1}{2}
\end{array}
\]
in korakom $h=0.1$. 
\item Izra�unaj absolutno in relativno napako (to�no re�itev dobi� z nastavkom
$y=x^{m}$). 
\end{enumerate}
\item Naj bosta 
\[
Q=\frac{1}{\sqrt{2}}\begin{bmatrix}1 & 0 & 1 & 0\\
0 & 1 & 0 & 1\\
1 & 0 & -1 & 0\\
0 & 1 & 0 & -1
\end{bmatrix}\quad R=\begin{bmatrix}2 & 0 & 1 & 2\\
0 & -3 & 0 & 1\\
0 & 0 & 2 & 1\\
0 & 0 & -1 & 2
\end{bmatrix}
\]
faktorja v realni Schurovi formi matrike $A=QRQ^{T}$.

\begin{enumerate}
\item Poi��i vse lastne vrednosti matrike in realne lastne vektorje za vsako
od realnih lastnih vrednosti matrike $A$. 
\item Ali poten�na metoda konvergira?\end{enumerate}
\end{enumerate}

\end{document}
