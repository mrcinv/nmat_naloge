%% LyX 2.0.3 created this file.  For more info, see http://www.lyx.org/.
%% Do not edit unless you really know what you are doing.
\documentclass[slovene]{article}
\usepackage[T1]{fontenc}
\usepackage[latin2]{inputenc}
\usepackage{amsmath}
\usepackage{esint}
\usepackage{babel}
\begin{document}

\title{Izpit iz Numeri�nih metod 2 za Finan�ne matematike}


\date{4. 9. 2013}
\maketitle
\begin{enumerate}
\item I��emo ni�lo funkcije 
\[
f(x)=x-2^{-x}
\]


\begin{enumerate}
\item Poka�i, da je ni�la na intervalu $[0,1]$ in jo poi��i na 2 decimalke
natan�no. 
\item Ni�lo poi��i z inverzno interpolacijo, tako da inverzno funkcijo $f^{-1}(x)$
interpolira� v to�kah $f(0),$ $f\left(\frac{1}{2}\right)$ in $f(1)$
s kvadratnim polinomom. Oceni napako in dobljeno ni�lo primerjaj z
re�itvijo iz to�ke $(a)$. 
\end{enumerate}
\item Izpeljite formulo za ra�unanje integrala oblike
\begin{equation}
\int_{0}^{4h}f(x)\mathrm{d}x=Af(h)+Bf(2h)+Cf(3h)+R(f),\label{eq:formula}
\end{equation}
kjer so $A,B,C$ realna �tevila in $R(f)$ napaka odvisna od funkcije
$f$. S formulo (\ref{eq:formula}) izra�unajte integral
\begin{equation}
\int_{0}^{2}\cos x\,{\rm dx}.\label{eq:integral}
\end{equation}
Kolik�na je relativna in absolutna napaka? Na podlagi formule (\ref{eq:formula})
izpeljite sestavljeno integracijsko pravilo in ugotovite, na ocenite
na koliko podintervalov je treba razdeliti interval $[0,2]$, da bo
pravilnih 5 decimalnih mest.
\item Za Hermitovo NDE 
\[
y''-2xy'+2y=0
\]
je ena od re�itev polinom $y_{1}(x)=x$. Naj bo $y_{2}(x)$ ena od
re�itev Hermitove DE, ki je neodvisna od $y_{1}$. Poi��i pribli�ke
za $y_{2}(0)$, $y_{2}(0.1)$ in $y_{2}(0.2)$ z metodo Runge-Kutta
s tabelo
\[
\begin{array}{c|cc}
0 & 0\\
1 & 1 & 0\\
\hline  & \frac{1}{2} & \frac{1}{2}
\end{array}.
\]
Za�etni pribli�ek izberite tako, da bo $y_{2}$ neodvisen od $y_{1}$.
\item Naj bo 
\[
A=\begin{bmatrix}20 & 0 & 1 & 2\\
0 & 1 & 1 & 2\\
0 & 0 & 1 & -2\\
1 & 0 & 0 & 2
\end{bmatrix}.
\]
Z Ger�gorinovimi krogi �im bolj natan�no ocenite lokacijo najve�je
lastne vrednosti. Z enim korakom poten�ne metode izra�unajte pribli�ek
za najve�jo lastno vrednost in pripadajo�i lastni vektor. Izra�unaj
lastne vrednosti matrike $A$ in primerjaj vrednosti s pribli�kom
poten�ne metode.\end{enumerate}

\end{document}
